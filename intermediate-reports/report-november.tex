\documentclass[12pt]{report}

\setlength{\parindent}{0cm}

\begin{document}
\pagestyle{myheadings}
\markright{Tussentijds verslag November -  Student: Axel Faes}
{\bf Titel eindwerk:} {\em Algebraic subtyping for algebraic effects and handlers}

\vspace{0.5cm}
{\bf Promotor:} prof. dr. ir. Tom Schrijvers


\vspace{0.5cm}
{\bf Begeleider:} Amr Hany Saleh

\vspace{1cm}
{\bf Korte situering en Doelstelling:} In mijn thesis ben ik een nieuw type systeem aan het uitwerken. Tijdens het verloop van de thesis worden ook de formele aspecten van dit systeem bestudeerd worden. Het type systeem is gefocussed op algebraïsche effecten en handlers en is een uitbreiding van een recent voorgesteld type systeem dat subtyping combineert met parametrisch polymorphisme. De doelstelling is om dit type systeem uit te werken en een proof-of-concept hiervan te implementeren en te testen.

\vspace{1cm}
{\bf Belangrijkste bestudeerde literatuur:}
\begin{itemize}
\item Matija Pretnar. An Introduction to Algebraic Effects and Handlers. 2015. url: http://www.eff-lang.org/handlers-tutorial.pdf.
\item Stephen Dolan and Alan Mycroft. “Polymorphism, Subtyping, and Type Inference in ML- sub”. In: Proceedings of the 44th ACM SIGPLAN Symposium on Principles of Programming Languages. POPL 2017. Paris, France: ACM, 2017, pp. 60–72. isbn: 978-1-4503-4660-3. doi: 10.1145/3009837.3009882. url: http://doi.acm.org/10.1145/3009837.3009882.
\item Benjamin C. Pierce. 2002. Types and Programming Languages (1st ed.). The MIT Press. url: http://dl.acm.org/citation.cfm?id=509043
\item Matija Pretnar. 2014. Inferring Algebraic Effects. Logical Methods in Computer Science 10, 3 (2014). https://doi.org/10.2168/LMCS-10(3: 21)2014
\end{itemize}

\vspace{1cm}
{\bf Geleverd werk (inclusief tijdsrapportering):} Ik heb het type systeem uitgewerkt. Meer specifiek heb ik de termen, types, subtyping regels, typing regels, type checking en constraint generatieregels uitgewerkt. Ik ben begonnen aan de implementatie en met de eerste bewijzen.

\vspace{1cm}
{\bf Belangrijkste resultaten:}
Het belganrijkste aspect komt neer op het uitgewerkte type systeem. Dit vormt de core van mijn thesis. Ook voor de implementatie is het handig om het algoritmisch gedeelte al uitgewerkt te hebben.


\vspace{1cm}
{\bf Belangrijkste moeilijkheden:}
Voornamelijk is het werk heel vlot verlopen tot nu toe. Door vorige ervaring, het honours programma, was ik al relatief bekend met de literatuur en algemene structuur van programmeertalen onderzoek. De moeilijkste aspecten was het afwerken van het type systeem. Alle details zijn belangrijk, dus er gaat wel wat tijd overheen voordat het type systeem volledig uitgewerkt was.

\vspace{1cm}
{\bf Gepland werk:}
De formele aspecten bewijzen en de implementatie vervolledigen zijn de eerstvolgende stappen. Hierna kan ik de implementatie gaan evalueren op basis van vergelijken met andere bestaande systemen. Als extra was ik aan het kijken naar een verdere toepassing van het type systeem. Dit zou gaan omtrent optimalisatie van algebraïsche effecten en handlers. Hieraan zijn enkele theoretische aspecten gekoppeld, namelijk het uitschrijven van de optimalisatie transformaties. Een groote deel zijn implementatie aspecten, namelijk het uitwerken van de optimalisaties voor niet-triviale voorbeelden.


\vspace{1cm}
{\bf Als ik verder werk zoals ik tot nu toe deed, dan denk ik 19/20 te verdienen op het einde.}
{\bf Ik plan mijn eindwerk af te geven in juni}


\end{document}
