\documentclass[master=cws,masteroption=ai]{kulemt}
\setup{title={Algebraic Subtyping for Algebraic Types and Effects},
  author={Axel Faes},
  promotor={Prof.\,dr.\,ir.\ Tom Schrijvers},
  assessor={assesors},
  assistant={Amr Hany Saleh}}
% The following \setup may be removed entirely if no filing card is wanted
\setup{filingcard,
  translatedtitle=,
  udc=,
  shortabstract={}}

% Choose the main text font (e.g., Latin Modern)
\setup{font=lm}

%% Some recommended packages.
\usepackage{booktabs}   %% For formal tables:
                        %% http://ctan.org/pkg/booktabs
\usepackage{subcaption} %% For complex figures with subfigures/subcaptions
                        %% http://ctan.org/pkg/subcaption

\usepackage{array}
\usepackage{mathpartir}
\usepackage{xspace}
\usepackage{stmaryrd}
\usepackage{listings}
\usepackage{newtxmath}

%% Tikz Needed packages
\usepackage{pgfplots}
\pgfplotsset{width=7cm,compat=1.8}
\usepackage{pgfplotstable}
\renewcommand*{\familydefault}{\sfdefault}

% Finally the hyperref package is used for pdf files.
% This can be commented out for printed versions.
\usepackage[pdfusetitle,colorlinks,plainpages=false]{hyperref}

\newcommand{\lang}{\textsc{Effy}\xspace}
\newcommand{\eff}{\textsc{Eff}\xspace}
\newcommand{\core}{\textsc{EffCore}\xspace}
\newcommand{\ocaml}{\textsc{OCaml}\xspace}

% Meta-syntax
\newcommand{\bnfis}{\mathrel{\;{:}{:}\!=}\;}
\newcommand{\bnfor}{\mathrel{\;|\;}}
\newcommand{\defeq}{\mathrel{\;\stackrel{\text{def}}{=}\;}}
\newcommand{\set}[1]{\{ #1 \}}

% General syntactic constructs
\newcommand{\kord}[1]{\mathtt{#1}}
\newcommand{\kop}[1]{\;\mathtt{#1}\;}
\newcommand{\kpre}[1]{\mathtt{#1}\;}
\newcommand{\kpost}[1]{\;\mathtt{#1}}

% Types
\newcommand{\type}[1]{\mathtt{#1}}
\newcommand{\boolty}{\type{bool}}
\newcommand{\intty}{\type{int}}
\newcommand{\hto}{\Rightarrow}
\renewcommand{\C}{\underline{C}}
\newcommand{\D}{\underline{D}}
\newcommand{\dirt}{\Delta}
\newcommand{\dirtend}{. (\textsc{dot})}
\newcommand{\sig}{\Sigma}
\newcommand{\allops}{\Omega}

% Expressions and computations
\newcommand{\funtyped}[3]{\kpre{fun} #1 \T #2 \mapsto #3}
\newcommand{\rectype}[1]{\mu \alpha . #1}
\newcommand{\letvar}{\textbf{\^{x}}}
\newcommand{\union}{\sqcup}
\newcommand{\intersection}{\sqcap}
\newcommand{\emptyrow}{\emptyset}
\newcommand{\polytype}[1]{\forall \bar{\alpha} . #1}

\newcommand{\call}[3]{{{#1}\,{#2}\,{#3}}}
\newcommand{\case}{\mathop{\text{\texttt{|}}}}
\newcommand{\cont}[2]{(#1.\,#2)}
\newcommand{\const}{\kord{k}}
\newcommand{\fls}{\kord{false}}
\newcommand{\fun}[1]{\lambda #1 .} %\kpre{fun} #1 \mapsto}
\newcommand{\handler}[1]{\{ #1 \}}
\newcommand{\conditional}[3]{\kpre{if} #1 \kop{then} #2 \kop{else} #3}
\newcommand{\letin}[1]{\kpre{let} #1 \kop{in}}
\newcommand{\doin}[1]{\kpre{do} #1 \kop{ ; }}
\newcommand{\letrecin}[1]{\kpre{let} \kpre{rec} #1 \kop{in}}
\newcommand{\op}{\kord{Op}}
\newcommand{\ops}{\mathcal{O}}
\newcommand{\ocs}{\mathit{ocs}}
\newcommand{\ocsnil}{\kord{nil}}
\newcommand{\tru}{\kord{true}}
\newcommand{\ret}{\kpre{return}}
\newcommand{\withhandle}[2]{\kpre{handle} #2 \kop{with} #1}
\newcommand{\pure}[1]{\kord{pure } #1  }
\newcommand{\longcases}{\call{\op_1}{y}{k} \mapsto c_{\op_1}, \ldots, \call{\op_n}{x}{k} \mapsto c_{\op_n}}
\newcommand{\shortcases}{[\call{\op}{y}{k} \mapsto c_\op]_{\op \in \ops}}
\newcommand{\longhand}[1][\ret x \mapsto c_r]{\handler{#1, \longcases}}
\newcommand{\shorthand}[1][\ret x \mapsto c_r]{\handler{#1, \shortcases}}
\newcommand{\shorthandelab}[1][\ret x \mapsto c'_r]{\handler{#1, \shortcaseselab}}
\newcommand{\shortcaseselab}{[\call{\op}{y}{k} \mapsto c'_\op]_{\op \in \ops}}

% Type-checking
\newcommand{\row}{\mathrel{;} R}
\newcommand{\ctx}{\Gamma}
\newcommand{\ctxm}{\Xi}
\newcommand{\ctxp}{\Pi}
\newcommand{\ent}{\vdash}
\newcommand{\entt}{\Vdash}
\newcommand{\T}{\mathrel{:}}
\newcommand{\E}{\mathrel{!}}
\newcommand{\covers}{\mathrel{/}}
\renewcommand{\le}{\leqslant}

% Operational semantics
\newcommand{\eval}{\Downarrow}
\newcommand{\hs}{\mathcal{H}}
\newcommand{\nil}{\emptyset}
\newcommand{\cons}{\mathbin{::}}
\newcommand{\hseval}[1][\hs]{\Downarrow_{#1}}
\newcommand{\getval}[1]{{#1}_{\kord{val}}}
\newcommand{\getop}[1]{{#1}_{\kord{op}}}

\newcommand{\todo}[1]{\textcolor{red}{\textsc{Todo:} #1}}

% Needed to use the files from spec
\let\subparagraph\paragraph
\let\paragraph\subsubsection
\let\subsubsection\subsection
\let\subsection\section
\let\section\chapter

\begin{document}

\begin{preface}
  I would like to thank everybody who kept me busy the last year,
  especially my promoter and my assistants. I would also like to thank the
  jury for reading the text.
\end{preface}

\tableofcontents*

\begin{abstract}
  Algebraic effects and handlers are a very active area of research. An important aspect is the development of an optimising compiler. \eff is an ML-style language with support for effects and forms the testbed for the optimising compiler. However, the type-\&-effect system of \eff is unsatisfactory. This is due to the lack of some elegant properties. It is also awkward to implement and use in practice.
\end{abstract}

\listoffigures
\listoftables

% Now comes the main text
\mainmatter

\section{Introduction}
The specification for a type-\&-effect system with algebraic subtyping for algebraic effects and handlers is given in this document. The formal properties of this system are studied in order to find which properties are satisfied compared to other type-\&-effect systems. The proposed type-\&-effect system builds on two very recent developments in the area of programming language theory.

\paragraph{Algebraic subtyping}
In his December 2016 PhD thesis, Stephen Dolan (University of Cambridge, UK), has presented a novel type system that combines subtyping and parametric polymorphism in a particulary attractive and elegant fashion. A cornerstone of his design are the algebraic properties that the subtyping relation should respect.

\paragraph{Algebraic effects and handlers}
These are a new formalism for formally modelling side-effects (e.g. mutable state or non-determinism) in programming languages, developed by Matija Pretnar (University of Ljubjana) and Gordon Plotkin (University of Edinburgh). This approach is gaining a lot of traction, not only as a formalism but also as a practical feature in actual programming languages (e.g. the Koka language developed by Microsoft Research). We are collaborating with Matija Pretnar on the efficient implementation of one such language, called Eff. Axel Faes has contributed to this collaboration during a project he did for the Honoursprogramme of the Faculty of Engineering Science.

\subsection{Motivation}
Algebraic effects and handlers benefit from a custom type-\&-effect system, a type system that also tracks which effects can happen in a program. Several such type-\&-effect systems have been proposed in the literature, but all are unsatisfactory. We attribute this to the lack of the elegant properties of Dolan's type system. Indeed the existing type-\&-effect systems are not only theoretically unsatisfactory, but they are also awkward to implement and use in practice.

\paragraph{Research questions}
\begin{itemize}
\item How can Dolan's elegant type system be extended with effect information?
\item Which properties are preserved and which aren't preserved?
\item What advantages are there to an type-\&-effect system based on Dolan's elegant type system?
\end{itemize}

\subsection{Goals}
The goal of this thesis is to derive a type-\&-effect system that extends Dolan's elegant type system with effect information. This type-\&-effect system should inherit Dolan's harmonious combination of subtyping (in our case induced by a lattice structure on the effect information) with parametric polymorphism and preserve all of its desirable properties (both low-level algebraic properties and high-level meta-theoretical properties like type soundness and the existence of principal types). Afterwards this type-\&-effect system The following approach is taken:
\begin{enumerate}
\item Study of the relevant literature and theoretical background.
\item Design of a type-\&-effect system derived from Dolan's, that integrates effects.
\item Proving the desirable properties of the proposed type-\&-effect system: type soundness, principal typing, ...
\item Time permitting: Design of a type inference algorithm that derives the principal types of programs without type annotations and proving its correctness.
\item Time permitting: Implementation of the algorithm and comparing it to other algorithms (such as row polymorphism based type-\&-effect systems).
\end{enumerate}

\subsection{Results}
Describe what the resulting product is and how it is useful or provides an advantage over other solutions.

\section{Background}
In this section, I will provide the background necessary to be able to read the text. This includes an introduction into programming languages (and programming language theory) and algebraic effect handlers 

Dolan's type system and \eff are discussed in further chapters and thus shouldn't need to be explained in this section.

\section{Related Work (Algebraic Subtyping)}
Subtyping is a partial order which is a reflexive transitive binary relation satisfying antisymmetry (subtyping rules). The subtyping order also forms a distributive lattice (equivalence rules). 



\section{Related Work (\eff)}
The type-\&-effect system that is used in \eff is based on subtyping and dirty types \cite{effectsystem}.

\subsection{Types and terms}

\paragraph{Terms}
Figure~\ref{fig:terms:eff} shows the two types of terms in \eff. There are values $v$ and computations $c$. Computations are terms that can contain effects. Effects are denoted as operations $Op$ which can be called.

\begin{figure}[!htb]
\begin{center}
\framebox{
\begin{minipage}{0.98\columnwidth}
\[\begin{array}{r@{~}c@{~}l@{\quad}l}
  \text{value}~v & \bnfis {} & x & \text{variable} \\ %\lambda\text{-variable} \\
    % & \bnfor & \letvar & \text{let-variable} \\
    % & \bnfor & \const & \text{constant} \\
    & \bnfor & \tru & \text{true} \\
    & \bnfor & \fls & \text{false} \\
    & \bnfor & \fun{x} c & \text{function} \\
    & \bnfor & \{ & \text{handler} \\
    & & \quad \ret x \mapsto c_r, & \quad\text{return case} \\
    & & \quad \shortcases & \quad\text{operation cases} \\
    & & \} & \\
  \text{comp}~c & \bnfis & v_1 \, v_2 & \text{application} \\
    & \bnfor & \doin{x \leftarrow c_1} c_2 & \text{sequencing} \\
    & \bnfor & \conditional{e}{c_1}{c_2} & \text{conditional} \\
    & \bnfor & \letrecin{f \, x = c_1} c_2 & \text{rec definition} \\
    & \bnfor & \ret v  & \text{returned val} \\
    & \bnfor & \op \, v & \text{operation call} \\
    & \bnfor & \withhandle{v}{c} & \text{handling}
\end{array}\]
\end{minipage}
}
\end{center}
\caption{Terms of \eff}\label{fig:terms:eff}
\end{figure}

\paragraph{Types}
Figure~\ref{fig:types:eff} shows the types of \eff. There are two main sorts of types. There are (pure) types $A, B$ and dirty types $\C, \D$. A dirty type is a pure type $A$ tagged with a finite set of operations $\dirt$, which we call dirt, that can be called. This finite set $\dirt$ is an over-approximation of the operations that are actually called. The type $\C \hto \D$ is used for handlers because a handler takes an input computation $\C$, handles the effects in this computation and outputs computation $\D$ as the result.

\begin{figure}[!htb]
\begin{center}
\framebox{
\begin{minipage}{0.98\columnwidth}
\[\begin{array}{r@{~}c@{~}l@{\quad}l}
  \text{(pure) type}~A, B & \bnfis {}
    % & \boolty \bnfor \intty & \text{basic types} \\
    & \boolty & \text{bool type} \\
    & \bnfor & A \to \C & \text{function type} \\
    & \bnfor & \C \hto \D & \text{handler type} \\
  \text{dirty type}~\C, \D & \bnfis {} & A \E \dirt \\
  \text{dirt}~\dirt & \bnfis {} &\set{\op_1, \dots, \op_n}
\end{array}\]
\end{minipage}
}
\end{center}
\caption{Types of \eff}\label{fig:types:eff}
\end{figure}

\subsection{Subtyping}
\eff uses a subtyping based system. Subtyping is a form of type polymorphism. Types can be related to eachother, being either subtypes or supertypes. Intuitively one could think about Java classes and inheritance in order to understand subtyping. There are some big differences between inheritance and subtyping, but from the principle of gaining an understanding of what subtyping entails, the relation between the two can be made. 

Let us take the subtyping judgement $\boolty \le \boolty$. This judgement is about reflexivity. It states that $\boolty$ is a subtype of itself. The subtyping judgement for the arrow type (functions) states that, if we have $A' \le A$ and $\C \le \C'$, then we also induce the natural subtyping judgement $A \to \C \le A' \to \C'$. This tells us that, if we have a function, the caller can always call that function with a type that is "more" and that function can always return "more" than what the caller expects. This can be easily visualised when the argument and return values are records. If a function requires a record with labels "x" and "y", the caller is allowed to call the function with a record containing more that just "x" and "y". A similar analogy can be made for the return values. Functions are contravariant in its argument types and covariant in its return types.

The dirty type $A \E \dirt$ is assigned to a computation returning values of type $A$ and potentially calling operations from the set $\dirt$. This set $\dirt$ is always an over-approximation of the actually called operations, and may safely be increased, inducing a natural subtyping judgement $A \E \dirt \le A \E \dirt'$ on dirty types where $\dirt'$ contains extra operations compared to $\dirt$. As dirty types can occur inside pure types, we also get a derived subtyping judgement on pure types. Both judgements are defined in Figure~\ref{fig:subtyping}. Observe that, as usual, subtyping is contravariant in the argument types of functions as well as handlers, and covariant in their return types. \cite{inferring}

\begin{figure}[!htb]
\begin{center}
\framebox{
\begin{minipage}{0.95\columnwidth}
\textbf{Subtyping}
\begin{mathpar}
  \inferrule[Sub-$\boolty$]{
  }{
    \boolty \le \boolty
  }

  \inferrule[Sub-$\to$]{
    A' \le A \\
    \C \le \C'
  }{
    A \to \C \le A' \to \C'
  }

  \inferrule[Sub-$\hto$]{
    \C' \le \C \\
    \D \le \D'
  }{
    \C \hto \D \le \C' \hto \D'
  }

  \inferrule[Sub-$\E$]{
    A \le A' \\
    \dirt \subseteq \dirt'
  }{
    A \E \dirt \le A' \E \dirt'
  }
\end{mathpar}
\end{minipage}
}
\end{center}
\caption{Subtyping for pure and dirty types of \eff}\label{fig:subtyping}
\end{figure}

\subsection{Typing rules}
Figure~\ref{fig:eff-typing:e} and figure~\ref{fig:eff-typing:c} defines the typing judgements for values and computations with respect to a standard typing context $\ctx$. This types context can contain $\epsilon$ or a variable with a type.

\paragraph{Values}
The rules for subtyping, variables, and functions are entirely standard.

A handler expression has type $A \E \dirt \cup \ops \hto B \E \dirt$ iff all branches (both the operation cases and the return case) have dirty type $B \E \dirt$ and the operation cases cover the set of operations $\ops$. Note that the intersection $\dirt \cap \ops$ is not necessarily empty. The handler deals with the operations $\ops$, but in the process may re-issue some of them (i.e., $\dirt \cap \ops$).

When typing operation cases, the given signature for the operation $(\op \T A_\op \to B_\op) \in \sig$ determines the type $A_\op$ of the parameter $x$ and the domain $B_\op$ of the continuation $k$. As our handlers are deep, the codomain of $k$ should be the same as the type $B \E \dirt$ of the cases. \cite{inferring}

\paragraph{Computations}
With the following exceptions, the typing judgement $\ctx \ent c : \C$ has a straightforward definition. The $\ret$ construct renders a value $v$ as a pure computation, i.e., with empty dirt. An operation invocation $\op\,v$ is typed according to the operation's signature, with the operation itself as its only operation. Finally, rule \textsc{With} shows that a handler with type $\C \hto \D$ transforms a computation with type $\C$ into a computation with type $\D$. \cite{inferring}

\begin{figure}[H]
\begin{center}
\framebox{
\begin{minipage}{0.95\columnwidth}
\[\begin{array}{r@{~}c@{~}l}
  \text{typing contexts}~\ctx & \bnfis {} & \epsilon \bnfor \ctx, x : A\\
\end{array}\]
\textbf{Expressions}
\begin{mathpar}
  \inferrule[SubVal]{
    \ctx \ent v \T A \\
    A \le A'
  }{
    \ctx \ent v \T A'
  }

  \inferrule[Var]{
    (x \T A) \in \ctx
  }{
    \ctx \ent x \T A
  }

  % \inferrule[Const]{
  %   (\const \T A) \in \sig
  % }{
  %   \ctx \ent \const \T A
  % }

  \inferrule[True]{
  }{
    \ctx \ent \tru \T bool
  }

  \inferrule[False]{
  }{
    \ctx \ent \fls \T bool
  }
  
  \inferrule[Fun]{
    \ctx, x \T A \ent c \T \C
  }{
    \ctx \ent \fun{x} c \T A \to \C
  }

  \inferrule[Hand]{
    \ctx, x \T A \ent c_r \T B \E \dirt \\
    \Big[
      (\op \T A_\op \to B_\op) \in \sig \qquad \\
      \ctx, y \T A_\op, k \T B_\op \to B \E \dirt \ent c_\op \T B \E \dirt
    \Big]_{\op \in \ops}
  }{
    \ctx \ent \shorthand \T \\ A \E \dirt \cup \ops \hto B \E \dirt
  }
\end{mathpar}
\end{minipage}
}
\end{center}
\caption{Typing of expressions in \eff}\label{fig:eff-typing:e}
\end{figure}

\begin{figure}[H]
  \begin{center}
  \framebox{
  \begin{minipage}{0.95\columnwidth}
  \[\begin{array}{r@{~}c@{~}l}
    \text{typing contexts}~\ctx & \bnfis {} & \epsilon \bnfor \ctx, x : A\\
  \end{array}\]
\textbf{Computations}
\begin{mathpar}
  \inferrule[SubComp]{
    \ctx \ent c \T \C \\
    \C \le \C'
  }{
    \ctx \ent c \T \C'
  }

  \inferrule[App]{
    \ctx \ent v_1 \T A \to \C \\
    \ctx \ent v_2 \T A
  }{
    \ctx \ent v_1 \, v_2 \T \C
  }

  \inferrule[Cond]{
    \ctx \ent v \T bool \\
    \ctx \ent c_1 \T \C \\
    \ctx \ent c_2 \T \C \\
  }{
    \ctx \ent \conditional{v}{c_1}{c_2} \T \C
  }

  \inferrule[LetRec]{
    \ctx, f \T A \to \C, x \T A \ent c_1 \T \C \\
    \ctx, f \T A \to \C \ent c_2 \T \D
  }{
    \ctx \ent \letrecin{f \, x = c_1} c_2 \T \D
  }

  \inferrule[Ret]{
    \ctx \ent v \T A
  }{
    \ctx \ent \ret v \T A \E \emptyset
  }

  \inferrule[Op]{
    (\op \T A \to B) \in \sig \\
    \ctx \ent v \T A
  }{
    \ctx \ent \op \, v \T B \E \{\op\}
  }

  \inferrule[Do]{
    \ctx \ent c_1 \T A \E \dirt \\
    \ctx, x \T A \ent c_2 \T B \E \dirt
  }{
    \ctx \ent \doin{x \leftarrow c_1} c_2 \T B \E \dirt
  }

  \inferrule[With]{
    \ctx \ent v \T \C \hto \D \\
    \ctx \ent c \T \C
  }{
    \ctx \ent \withhandle{v}{c} \T \D
  }
\end{mathpar}
\end{minipage}
}
\end{center}
\caption{Typing of computations in \eff}\label{fig:eff-typing:c}
\end{figure}

\subsection{Type Inference}\label{type-inference-explain}
Type inference is the process where types are automatically inferred by the compiler. Types rules are used as a blueprint for type inference. Every typing rule indicates a situation a program can be in at any point in time. Thus, for every typing rule, there has to be a type inference rule. 

In the case of a subtyping based system, contraint-based type inference rules are used. The specific rules for \eff are not fully given as they are not required for the work in this thesis. The idea behind constraint-based type inference rules is that, in each rule, constraints can be made. In case of a subtyping based system, these constraints are subtyping constraints between two types. 

In figure~\ref{fig:inference:eff}, the type inference for function specialization can be seen. We have two expressions $v_1$ and $v_2$ with types $A_1$ and $A_2$. The application produces some type $\alpha \E \delta$. In order to link the types of the two expressions to the produced type, a subtyping constraint is used. The constraint $A_1 \le A_2 \to (\alpha \E \delta)$ indicates that $A_1$ has to be a subtype of a function type $A_2 \to (\alpha \E \delta)$. 

The reader may wonder what happens when the subtyping constraint is changed into an equality constraint $A_1 = A_2 \to (\alpha \E \delta)$. If every subtyping relation is changed into an equality relation, including for all relations in the subtyping rules in figure~\ref{fig:subtyping}, than we have changed the subtyping system into a Hindley-Milner system. The Hindley-Milner system is less expressive than the subtyping system. This makes sense as an equation with subtyping $\le$ allows for more solutions than using equality $=$.

For every typing rule, there is a type inference rule. At the end of the of applying all the rules, $\ctrs$ contains a lot of constraints. These constraints are solved as much as possible using substitution techniques. Subtyping does not allow for all constraints to be completely solved. In contrast, the Hindley-Milner system can solve all constraints with substitution techniques.

\begin{figure}[H]
\begin{center}
  \begin{framed}
  \begin{minipage}[t]{0.95\columnwidth}
  \[\begin{array}{r@{~}c@{~}l}
      \text{typing contexts}~\ctx & \bnfis {} & \epsilon \bnfor \ctx, x : A\\
      \end{array}\]
  \textbf{Computations}
      \begin{mathpar}
      \inferrule[App]{
          \ctx \prinent v_1 \T A_1 \\
          \ctx \prinent v_2 \T A_2
      }{
          \ctx \prinent v_1 \, v_2 \T \alpha \E \delta
      } \ctrs = \ctrs \cup (A_1 \le A_2 \to (\alpha \E \delta))
      \end{mathpar}
  \end{minipage}
  \end{framed}
  \end{center}
  \caption{Type inference rule for function application for \eff}\label{fig:inference:eff}
  \end{figure}
  

\section{Core Language (\core)}

\core is a language with row-based effects, intersection and union types and effects and is subtyping based. 

Define your problem very clearly. Provide a formal definition if possible, using mathematical definitions.

\subsection{Types and terms}

\paragraph{Terms}
Figure~\ref{fig:terms:core} shows the two types of terms in \core. Just like in \eff. there are values $v$ and computations $c$. Computations are terms that can contain effects. Effects are denoted as operations $Op$ which can be called. The only change compared to \eff is that \core makes a distinction between let-bound variables and lambda-bound variables. This distinction was introduced by Dolan in order to simplify the algebraic subtyping approach \cite{mlsub}. By making this distinction, a distinction can be made between monomorphic variables (lambda-bound) and polymorphic variables (let-bound) at the term level.

\begin{figure}[!htb]
\begin{center}
\framebox{
\begin{minipage}{0.98\columnwidth}
\[\begin{array}{r@{~}c@{~}l@{\quad}l}
  \text{value}~v & \bnfis {} & x & \lambda\text{-variable} \\
    & \bnfor & \letvar & \text{let-variable} \\
    & \bnfor & \tru & \text{true} \\
    & \bnfor & \fls & \text{false} \\
    & \bnfor & \fun{x}c & \text{function} \\
    & \bnfor & \{ & \text{handler} \\
    & & \quad \ret x \mapsto c_r, & \quad\text{return case} \\
    & & \quad \shortcases & \quad\text{operation cases} \\
    & & \} & \\
  \text{comp}~c & \bnfis & v_1 \, v_2 & \text{application} \\
    & \bnfor & \doin{\letvar = c_1} c_2 & \text{sequencing} \\
    & \bnfor & \letin{\letvar = v} c & \text{let} \\
    & \bnfor & \conditional{e}{c_1}{c_2} & \text{conditional} \\
    & \bnfor & \ret v  & \text{returned val} \\
    & \bnfor & \op \, v & \text{operation call} \\
    & \bnfor & \withhandle{v}{c} & \text{handling}
\end{array}\]
\end{minipage}
}
\end{center}
\caption{Terms of \core}\label{fig:terms:core}
\end{figure}

\paragraph{Types}
Figure~\ref{fig:types:core} shows the types of \core. There are, like in \eff, two main sorts of types. There are (pure) types $A, B$ and dirty types $\C, \D$. A dirty type is a pure type $A$ tagged with a finite set of operations $\dirt$, which we call dirt, that can be called. It can also be an union or intersection of dirty types. In further sections, the relations between dirty intersections or unions and pure intersections or unions are explained. The finite set $\dirt$ is an over-approximation of the operations that are actually called. Row variables are introduced as well as intersection and unions. The $\dirtend$ is used to close rows that do not end with a row variable. The type $\C \hto \D$ is used for handlers because a handler takes an input computation $\C$, handles the effects in this computation and outputs computation $\D$ as the result. \cite{handling}

However, now the effects of the algebraic subtyping approach become apparant. Different types are added in order to support the subtyping. These are a type variables, recursive type, top, bottom, intersection and union \cite{mlsub}. The novel element here is the combination of the algebraic effects and algebraic subtyping. There needs to be a way to bring the dirts into the algebraic subtyping framework. Since the recursive element is handled at the term level, there is no need for recursive dirts. Aside from this and the lack of a function and handler type, the dirts mirror the types.

\begin{figure}[!htb]
\begin{center}
\framebox{
\begin{minipage}{0.98\columnwidth}
\[\begin{array}{r@{~}c@{~}l@{\quad}l}
  \text{typing contexts}~\ctx & \bnfis {} & \epsilon \bnfor \ctx, x : A \bnfor \ctx, \letvar : \polytype{B}\\
  \text{monomorphic typing contexts}~\ctxm & \bnfis {} & \epsilon \bnfor \ctxm, x : A\\
  \text{polymorphic typing contexts}~\ctxp & \bnfis {} & \epsilon \bnfor \ctxp, \letvar : [\ctxm]A\\
  \text{(pure) type}~A, B & \bnfis {}
    & \boolty & \text{bool type} \\
    & \bnfor & A \to \C & \text{function type} \\
    & \bnfor & \C \hto \D & \text{handler type} \\
    & \bnfor & \alpha & \text{type variable} \\
    & \bnfor & \rectype{A} & \text{recursive type} \\
    & \bnfor & \top & \text{top} \\
    & \bnfor & \bot & \text{bottom} \\
    & \bnfor & A \intersection B & \text{intersection} \\
    & \bnfor & A \union B & \text{union} \\
  \text{dirty type}~\C, \D & \bnfis {} & A \E \dirt \\
    
  \text{dirt}~\dirt & \bnfis {} & \op & \text{operation} \\
    & \bnfor & \delta & \text{dirt variable} \\
    & \bnfor & \emptyrow & \text{empty dirt} \\
    & \bnfor & \dirt_1 \intersection \dirt_2 & \text{intersection} \\
    & \bnfor & \dirt_1 \union \dirt_2 & \text{union} \\
  \text{All operations}~\allops & \bnfis {} & \bigsqcup \op_i | \op_i \in \sig \\
\end{array}\]
\end{minipage}
}
\end{center}
\caption{Types of \core}\label{fig:types:core}
\end{figure}

\subsection{Type system}

\todo{explain equivalance rules}

\begin{figure}[!htb]
\begin{center}
\begin{framed}
\begin{minipage}[t]{0.95\columnwidth}
\begin{mathpar}    
    \inferrule[]{}{
        A_1 \le A_2 \leftrightarrow A_1 \union A_2 \equiv A_2
    }\\

    \inferrule[]{}{
        A_1 \le A_2 \leftrightarrow A_1 \equiv A_1 \intersection A_2
    }\\

    % \inferrule[]{}{
    %     \op \le \delta_1 \leftrightarrow \delta_2 = \op \union \delta_1, \op \union \delta_2 \equiv \delta_2
    % }\\

    \inferrule[]{}{
        \dirt_1 \le \dirt_2 \leftrightarrow \dirt_1 \union \dirt_2 \equiv \dirt_2
    }\\

    \inferrule[]{}{
        \dirt_1 \le \dirt_2 \leftrightarrow \dirt_1 \equiv \dirt_1 \intersection \dirt_2
    }\\

    \inferrule[]{}{
        \C_1 \le \C_2 \leftrightarrow \C_1 \union \C_2 \equiv \C_2
    }\\

    \inferrule[]{}{
        \C_1 \le \C_2 \leftrightarrow \C_1 \equiv \C_1 \intersection \C_2
    }
\end{mathpar}
\end{minipage}
\end{framed}
\end{center}
\caption{Relationship between Equivalence and Subtyping}\label{fig:core-relation}
\end{figure}


\begin{figure}[!htb]
\begin{center}
\begin{framed}
% \framebox{
\begin{minipage}[t]{0.475\columnwidth}
% \textbf{Equations of distributive lattices for types}
\begin{mathpar}
    \inferrule[]{}{
      A \union A \equiv A
    }\\
    
    \inferrule[]{}{
      A_1 \union A_2 \equiv A_2 \union A_1
    }\\

    \inferrule[]{}{
      A_1 \union (A_2 \union A_3) \equiv (A_1 \union A_2) \union A_3
    }\\

    \inferrule[]{}{
      A_1 \union (A_1 \intersection A_2) \equiv A_1
    }\\

    \inferrule[]{}{
      \bot \union A \equiv A
    }\\

    \inferrule[]{}{
      \top \union A \equiv \top
    }
\end{mathpar}
\end{minipage}
\begin{minipage}[t]{0.475\columnwidth}
\begin{mathpar}    
    \inferrule[]{}{
        A \intersection A \equiv A
    }\\

    \inferrule[]{}{
        A_1 \intersection A_2 \equiv A_2 \intersection A_1
    }\\

    \inferrule[]{}{
        A_1 \intersection (A_2 \intersection A_3) \equiv (A_1 \intersection A_2) \intersection A_3
    }\\

    \inferrule[]{}{
        A_1 \intersection (A_1 \union A_2) \equiv A_1
    }\\

    \inferrule[]{}{
        \bot \intersection A \equiv \bot
    }\\

    \inferrule[]{}{
        \top \intersection A \equiv A
    }
\end{mathpar}
\end{minipage}
\begin{minipage}[t]{0.95\columnwidth}
\begin{mathpar}    
    \inferrule[]{}{
        A_1 \union (A_2 \intersection A_3) \equiv (A_1 \union A_2) \intersection (A_1 \union A_3)
    }\\

    \inferrule[]{}{
        A_1 \intersection (A_2 \union A_3) \equiv (A_1 \intersection A_2) \union (A_1 \intersection A_3)
    }
\end{mathpar}
\end{minipage}
% }
\end{framed}
\end{center}
\caption{Equations of distributive lattices for types}\label{fig:core-equations-types}
\end{figure}

\begin{figure}[!htb]
\begin{center}
\begin{framed}
\begin{minipage}[t]{0.95\columnwidth}
\begin{mathpar}    
    \inferrule[]{}{
        (A_1 \to \C_1) \union (A_2 \to \C_2) \equiv (A_1 \intersection A_2) \to (\C_1 \union \C_2)
    }\\

    \inferrule[]{}{
        (A_1 \to \C_1) \intersection (A_2 \to \C_2) \equiv (A_1 \union A_2) \to (\C_1 \intersection \C_2)
    }\\

    \inferrule[]{}{
        (A_1 \hto \C_1) \union (A_2 \hto \C_2) \equiv (A_1 \intersection A_2) \hto (\C_1 \union \C_2)
    }\\

    \inferrule[]{}{
        (A_1 \hto \C_1) \intersection (A_2 \hto \C_2) \equiv (A_1 \union A_2) \hto (\C_1 \intersection \C_2)
    }\\
    
    \inferrule[]{}{
        (\C_1 \union \C_2) \equiv (A_1 \E \dirt_1 \union A_2 \E \dirt_2) \equiv (A_1 \union A_2) \E (\dirt_1 \union \dirt_2)
    }\\

    \inferrule[]{}{
        (\C_1 \intersection \C_2) \equiv (A_1 \E \dirt_1 \intersection A_2 \E \dirt_2) \equiv (A_1 \intersection A_2) \E (\dirt_1 \intersection \dirt_2)
    }
\end{mathpar}
\end{minipage}
\end{framed}
\end{center}
\caption{Equations for function, handler and dirty types}\label{fig:core-equations-other-types}
\end{figure}

\begin{figure}[!htb]
\begin{center}
\begin{framed}
% \framebox{
\begin{minipage}[t]{0.475\columnwidth}
\begin{mathpar}
    \inferrule[]{}{
        \dirt \union \dirt \equiv \dirt
    }\\
    
    \inferrule[]{}{
        \dirt_1 \union \dirt_2 \equiv \dirt_2 \union \dirt_1
    }\\

    \inferrule[]{}{
        \dirt_1 \union (\dirt_2 \union \dirt_3) \equiv (\dirt_1 \union \dirt_2) \union \dirt_3
    }\\

    \inferrule[]{}{
        \dirt_1 \union (\dirt_1 \intersection \dirt_2) \equiv \dirt_1
    }\\

    \inferrule[]{}{
        \emptyrow \union \dirt \equiv \dirt
    }\\

    \inferrule[]{}{
        \allops \union \dirt \equiv \allops
    }
\end{mathpar}
\end{minipage}
\begin{minipage}[t]{0.475\columnwidth}
\begin{mathpar}    
    \inferrule[]{}{
        \dirt \intersection \dirt \equiv \dirt
    }\\

    \inferrule[]{}{
        \dirt_1 \intersection \dirt_2 \equiv \dirt_2 \intersection \dirt_1
    }\\

    \inferrule[]{}{
        \dirt_1 \intersection (\dirt_2 \intersection \dirt_3) \equiv (\dirt_1 \intersection \dirt_2) \intersection \dirt_3
    }\\

    \inferrule[]{}{
        \dirt_1 \intersection (\dirt_1 \union \dirt_2) \equiv \dirt_1
    }\\

    \inferrule[]{}{
        \emptyrow \intersection \dirt \equiv \emptyrow
    }\\

    \inferrule[]{}{
        \allops \intersection \dirt \equiv \dirt
    }
\end{mathpar}
\end{minipage}
\begin{minipage}[t]{0.95\columnwidth}
\begin{mathpar}    
    \inferrule[]{}{
        \dirt_1 \union (\dirt_2 \intersection \dirt_3) \equiv (\dirt_1 \union \dirt_2) \intersection (\dirt_1 \union \dirt_3)
    }\\

    \inferrule[]{}{
        \dirt_1 \intersection (\dirt_2 \union \dirt_3) \equiv (\dirt_1 \intersection \dirt_2) \union (\dirt_1 \intersection \dirt_3)
    }
\end{mathpar}
\end{minipage}
% }
\end{framed}
\end{center}
\caption{Equations of distributive lattices for dirts}\label{fig:core-equations-dirts}
\end{figure}
% \subsection{Subtyping rules}
% The subtyping rules are given in Figure~\ref{fig:core-subtyping} and Figure~\ref{fig:core-subtyping-dirt}. Figure~\ref{fig:core-subtyping} contains all subtyping rules related to types. Figure~\ref{fig:core-subtyping-dirt} contains the subtyping rules that govern the dirts.

% The dirty type $A \E \dirt$ is assigned to a computation returning values of type $A$ and potentially calling operations from the set $\dirt$. This set $\dirt$ is always an over-approximation of the actually called operations, and may safely be increased, inducing a natural subtyping judgement $A \E \dirt \leq A \E \dirt'$ on dirty types. As dirty types can occur inside pure types, we also get a derived subtyping judgement on pure types. Observe that, as usual, subtyping is contravariant in the argument types of functions and handlers, and covariant in their return types.

% \paragraph{Dirt intersection and union}
% There are several possible methods to compute the dirt intersection and union. If row variables were to be disregarded, dirt union and intersection could be defined as set union and intersection. This methods allows unions and intersections to be eliminated. This has an advantage, eliminating unions and intersections simplifies the effect system. However, we cannot disregard row variables.

% Thus, set union and intersection cannot simply be used. It would be possible to define $\delta_1 \union \delta_2$ and $\delta_1 \intersection \delta_2$. Using these, it is possible to use a form of set union and intersection. The following union $\{Op_1, ..., Op_n, \delta_1\} \union \{Op_{n+1}, ..., Op_{n+m}, \delta_2\}$ could be defined as $\{Op_1, ..., Op_n, Op_{n+1}, ..., Op_{n+m}, (\delta_1 \union \delta_2)\}$. A similar construction can be used for intersection. This simplifies the subtyping rules since the more complicated aspects are enclosed within the row variables. The equivalence rules are defined in Figure~\ref{fig:core-equivalence}.



\begin{figure}[!htb]
\begin{center}
\framebox{
\begin{minipage}{0.95\columnwidth}
\textbf{Subtyping of dirts}
\begin{mathpar}





  \inferrule[Sub-$\E$-Row]{
  }{
     \{Op_1, ..., Op_n, .\} \le \{Op_1, ..., Op_n, \delta\}
  }

  


  \inferrule[Sub-$\E$-Row-Row]{
    n \ge 0 \\
    m \ge 0 \\
    p \ge 0 \\
    \{Op_1, ..., Op_{n}, Op_{n+m+1}, ..., Op_{n+m+p}, \delta_1\} \le \\ \{Op_1, ..., Op_n, Op_{n+1}, ..., Op_{n+m}, \delta_2\}
  }{
    \{\delta_1\} \le \{Op_{n+1}, ..., Op_{n+m}, \delta_2\} \\
    \{\delta_2\} = \{Op_{n+m}, ..., Op_{n+m+p}, \delta_3\}
  }

  \inferrule[Sub-$\E$-Dot-Row]{
    n \ge 0 \\
    m \ge 0 \\
    p \ge 0 \\
    \{Op_1, ..., Op_{n}, Op_{n+m+1}, ..., Op_{n+m+p}, .\} \le \\ \{Op_1, ..., Op_n, Op_{n+1}, ..., Op_{n+m}, \delta_2\}
  }{
    \emptyrow \le \{Op_{n+1}, ..., Op_{n+m}, \delta_2\} \\
    \{\delta_2\} = \{Op_{n+m}, ..., Op_{n+m+p}, \delta_3\}
  }

  \inferrule[Sub-$\E$-Row-Dot]{
    n \ge 0 \\
    m \ge 0 \\
    \{Op_1, ..., Op_n, \delta_1\} \le \{Op_1, ..., Op_n, Op_{n+1}, Op_{n+m}, .\}
  }{
    \{\delta_1\} \le \{Op_{n+1}, Op_{n+m}, .\}
  }

  \inferrule[Sub-$\E$-Dot-Dot]{
    n \ge 0 \\
    m \ge 0 \\
    \{Op_1, ..., Op_n, .\} \le \{Op_1, ..., Op_n, Op_{n+1}, ..., Op_{n+m}, .\}
  }{
    \emptyrow \le \{Op_{n+1}, Op_{n+m}, .\}
  }

\end{mathpar}
\end{minipage}
}
\end{center}
\caption{Subtyping for dirts of \core}\label{fig:core-subtyping-dirt}
\end{figure}


\subsection{Typing rules}
Figure~\ref{fig:core-typing} defines the typing judgements for values and computations with respect to a standard typing context $\ctx$.

\paragraph{Values}
The rules for subtyping, variables, type abstraction, type application and functions are entirely standard. For constants we assume a signature $\sig$ that assigns a type~$A$ to each constant~$\const$, which we write as $(\const \T A) \in \sig$.

A handler expression has type $A \E \dirt \cup \ops \hto B \E \dirt$ iff all branches (both the operation cases and the return case) have dirty type $B \E \dirt$ and the operation cases cover the set of operations $\ops$. Note that the intersection $\dirt \cap \ops$ is not necessarily empty (with $\cap$ being the intersection of the operations, not to be confused with the $\intersection$ type). The handler deals with the operations $\ops$, but in the process may re-issue some of them (i.e., $\dirt \cap \ops$).

When typing operation cases, the given signature for the operation $(\op \T A_\op \to B_\op) \in \sig$ determines the type $A_\op$ of the parameter $x$ and the domain $B_\op$ of the continuation $k$. As our handlers are deep, the codomain of $k$ should be the same as the type $B \E \dirt$ of the cases.

\paragraph{Computations}
With the following exceptions, the typing judgement $\ctx \ent c : \C$ has a straightforward definition. The $\ret$ construct renders a value $v$ as a pure computation, i.e., with empty dirt. In this case, this is defined as a set with the $\dirtend$ as the only element. An operation invocation $\op\,v$ is typed according to the operation's signature, with the operation itself as its only operation. Finally, rule \textsc{With} shows that a handler with type $\C \hto \D$ transforms a computation with type $\C$ into a computation with type $\D$.

\begin{figure}[!htb]
\begin{center}
\framebox{
\begin{minipage}{0.95\columnwidth}
\[\begin{array}{r@{~}c@{~}l}
  \text{typing contexts}~\ctx & \bnfis {} & \epsilon \bnfor \ctx, x : A \bnfor \ctx, \letvar : \polytype{B}\\
\end{array}\]
\textbf{Expressions}
\begin{mathpar}
  \inferrule[SubVal]{
    \ctx \ent v \T A \\
    A \le B
  }{
    \ctx \ent v \T B
  }

  \inferrule[Var-$\lambda$]{
    (x \T A) \in \ctx
  }{
    \ctx \ent x \T A
  }

  \inferrule[Var-$\forall$]{
    (\letvar \T \polytype{A}) \in \ctx
  }{
    \ctx \ent \letvar \T A[\bar{A}/\bar{\alpha}]
  }

  \inferrule[True]{
  }{
    \ctx \ent \tru \T bool
  }

  \inferrule[False]{
  }{
    \ctx \ent \fls \T bool
  }

  \inferrule[Fun]{
    \ctx, x \T A \ent c \T \C
  }{
    \ctx \ent \fun{x}c \T A \to \C
  }

  \inferrule[Hand]{
    \ctx, x \T A \ent c_r \T B \E \dirt \\
    \Big[
      (\op \T A_\op \to B_\op) \in \sig \qquad \\
      \ctx, y \T A_\op, k \T B_\op \to B \E \dirt \ent c_\op \T B \E \dirt
    \Big]_{\op \in \ops}
  }{
    \ctx \ent \shorthand \T \\ A \E \dirt \union \ops \hto B \E \dirt
  }

\end{mathpar}
\textbf{Computations}
\begin{mathpar}
  \inferrule[SubComp]{
    \ctx \ent c \T \C \\
    \C \le \D
  }{
    \ctx \ent c \T \D
  }

  \inferrule[App]{
    \ctx \ent v_1 \T A \to \C \\
    \ctx \ent v_2 \T A
  }{
    \ctx \ent v_1 \, v_2 \T \C
  }

  \inferrule[Cond]{
    \ctx \ent v \T bool \\
    \ctx \ent c_1 \T \C \\
    \ctx \ent c_2 \T \C \\
  }{
    \ctx \ent \conditional{v}{c_1}{c_2} \T \C
  }

  \inferrule[Ret]{
    \ctx \ent v \T A
  }{
    \ctx \ent \ret v \T A \E \emptyrow
  }

  \inferrule[Op]{
    (\op \T A \to B) \in \sig \\
    \ctx \ent v \T A
  }{
    \ctx \ent \op \, v \T B \E \op
  }

  \inferrule[Let]{
    \ctx \ent v \T A \\
    \ctx, \letvar \T \polytype{A} \ent c \T B \E \dirt \\
    \alpha \not\in FTV(\ctx)
  }{
    \ctx \ent \letin{\letvar = v} c \T B \E \dirt
  }

  \inferrule[Do]{
    \ctx \ent c_1 \T A \E \dirt \\
    \ctx, \letvar \T \polytype{A} \ent c_2 \T B \E \dirt \\
    \alpha \not\in FTV(\ctx)
  }{
    \ctx \ent \doin{\letvar = c_1} c_2 \T B \E \dirt
  }

  \inferrule[With]{
    \ctx \ent v \T \C \hto \D \\
    \ctx \ent c \T \C
  }{
    \ctx \ent \withhandle{v}{c} \T \D
  }

\end{mathpar}
\end{minipage}
}
\end{center}
\caption{Typing of \core}\label{fig:core-typing}
\end{figure}

\subsection{Reformulated typing rules}

\begin{figure}[!htb]
    \begin{center}
    \framebox{
    \begin{minipage}{0.95\columnwidth}
    \[\begin{array}{r@{~}c@{~}l}
      \text{monomorphic typing contexts}~\ctxm & \bnfis {} & \epsilon \bnfor \ctxm, x : A\\
      \text{polymorphic typing contexts}~\ctxp & \bnfis {} & \epsilon \bnfor \ctxp, \letvar : [\ctxm]A \bnfor \ctxp, \letvar : [\ctxm]\C\\
    \end{array}\]
    \textbf{Expressions}
    \begin{mathpar}
      \inferrule[Sub-Val]{
        \ctxp \entt v \T [\ctxm_1]A_1 \\
        [\ctxm_1]A_1 \le^\forall [\ctxm_2]A_2
      }{
        \ctxp \entt v \T [\ctxm_2]A_2
      }
    
      \inferrule[Var-$\lambda$]{
      }{
        \ctxp \entt x \T [x : A]A
      }
    
      \inferrule[Var-$\forall$]{
        (\letvar \T [\ctxm]A) \in \ctxp
      }{
        \ctxp \entt \letvar \T [\ctxm]A
      }    
    
      \inferrule[True]{
      }{
        \ctxp \entt \tru \T []bool
      }
    
      \inferrule[False]{
      }{
        \ctxp \entt \fls \T []bool
      }
    
      \inferrule[Fun]{
        \ctxp \entt c \T [\ctxm, x \T A]\C
      }{
        \ctxp \entt \fun{x}c \T [\ctxm]A \to \C
      }
    
      \inferrule[Hand]{
        \ctx, x \T A \ent c_r \T B \E \dirt \\
        \Big[
          (\op \T A_\op \to B_\op) \in \sig \qquad \\
          \ctx, x \T A_\op, k \T B_\op \to B \E \dirt \ent c_\op \T B \E \dirt
        \Big]_{\op \in \ops}
      }{
        \ctx \ent \shorthand \T \\ A \E \dirt \cup \ops \hto B \E \dirt
      }
    
    \end{mathpar}
    \textbf{Computations}
    \begin{mathpar}
      \inferrule[Sub-Comp]{
        \ctxp \entt c \T [\ctxm_1]\C_1 \\
        [\ctxm_1]\C_1 \le^\forall [\ctxm_2]\C_2
      }{
        \ctxp \entt c \T [\ctxm_2]\C_2
      }
    
      \inferrule[App]{
        \ctxp \entt v_1 \T [\ctxm]A \to \C \\
        \ctxp \entt v_2 \T [\ctxm]A
      }{
        \ctxp \entt v_1 \, v_2 \T [\ctxm]\C
      }
    
      \inferrule[Cond]{
        \ctxp \entt v \T [\ctxm]bool \\
        \ctxp \entt c_1 \T [\ctxm]\C \\
        \ctxp \entt c_2 \T [\ctxm]\C \\
      }{
        \ctxp \entt \conditional{v}{c_1}{c_2} \T [\ctxm](\C)
      }
    
      \inferrule[Ret]{
        \ctxp \entt v \T [\ctxm]A
      }{
        \ctxp \entt \ret v \T [\ctxm]A \E \emptyrow
      }
    
      \inferrule[Op]{
        (\op \T A \to B) \in \sig \\
        \ctxp \entt v \T [\ctxm]A \\
        \C \T [\ctxm]B \E \{\op \row\}
      }{
        \ctxp \entt \op \, v \T [\ctxm]\C
      }
    
      \inferrule[Let]{
        \ctxp \entt c_1 \T [\ctxm_1]A \E \dirt \\
        \ctxp, x \T [\ctxm_1]A \entt c_2 \T [\ctxm_2]B \E \dirt
      }{
        \ctx \ent \letin{\letvar = c_1} c_2 \T [\ctxm_1 \intersection \ctxm_2]B \E \dirt
      }
    
      \inferrule[With]{
        \ctxp \entt v \T [\ctxm]\C \hto \D \\
        \ctxp \entt c \T [\ctxm]\C
      }{
        \ctxp \entt \withhandle{v}{c} \T [\ctxm]\D
      }
    
    \end{mathpar}
    \end{minipage}
    }
    \end{center}
    \caption{Reformulated typing rules of \core}\label{fig:core-reform-typing}
    \end{figure}
    
\subsection{Semantics}

\section{Type Inference}

\subsection{Elaboration of \eff into \core}
% The elaboration show how the source language can be transformed into \core.

% \begin{figure}[!htb]
% \begin{center}
% \framebox{
% \begin{minipage}{0.95\columnwidth}
% \[\begin{array}{r@{~}c@{~}l}
%   \text{typing contexts}~\ctx & \bnfis {} & \epsilon \bnfor \ctx, x : A \bnfor \ctx,  x : \polytype{B}\\
% \end{array}\]
% \textbf{Expressions}
% \begin{mathpar}
%   \inferrule[Val]{
%     \ctx \ent v \T A \leadsto v'\\
%     \ent_c A \le B
%   }{
%     \ctx \ent v \T B \leadsto v'
%   }

%   \inferrule[Var]{
%     (x \T S) \in \ctx \\
%     S = \forall \bar{\alpha} . A
%   }{
%     \ctx \ent x \T A[\bar{S}/\bar{\alpha}] \leadsto x \, \bar{S}
%   }

%   \inferrule[Const]{
%     (\const \T A) \in \sig
%   }{
%     \ctx \ent \const \T A \leadsto \const'
%   }

%   \inferrule[Fun]{
%     \ctx, x \T A \ent c \T \C \leadsto c'
%   }{
%     \ctx \ent \fun{x} c \T A \to \C \leadsto \funtyped{x}{A}{c'} \T A \to \C
%   }

%   \inferrule[Hand]{
%     \ctx, x \T A \ent c_r \T B \E \dirt \\
%     \Big[
%       (\op \T A_\op \to B_\op) \in \sig \qquad \\
%       \ctx, x \T A_\op, k \T B_\op \to B \E \dirt \ent c_\op \T B \E \dirt
%     \Big]_{\op \in \ops}
%   }{
%     \ctx \ent \shorthand \T \\ A \E \dirt \cup \ops \hto B \E \dirt \\ \leadsto \shorthand \T \\ A \E \dirt \cup \ops \hto B \E \dirt
%   }

% \end{mathpar}
% \end{minipage}
% }
% \end{center}
% \caption{Elaboration of source to core language: expressions}\label{fig:elaboration:exp}
% \end{figure}


% \begin{figure}[!htb]
% \begin{center}
% \framebox{
% \begin{minipage}{0.95\columnwidth}
% \[\begin{array}{r@{~}c@{~}l}
%   \text{typing contexts}~\ctx & \bnfis {} & \epsilon \bnfor \ctx, x : A, x : \polytype{B}\\
% \end{array}\]
% \textbf{Computations}
% \begin{mathpar}
%   \inferrule[Comp]{
%     \ctx \ent c \T \C \leadsto c'\\
%     \C \le \D
%   }{
%     \ctx \ent c \T \D \leadsto c'
%   }

%   \inferrule[App]{
%     \ctx \ent v_1 \T A \to \C \leadsto v'_1 \\
%     \ctx \ent v_2 \T A \leadsto v'_2
%   }{
%     \ctx \ent v_1 \, v_2 \T \C \leadsto v'_1 \, v'_2 \T \C
%   }

%  \inferrule[LetRec]{
%     \ctx, f \T A \to \C, x \T A \ent c_1 \T \C \leadsto c'_1 s\\
%     \ctx, f \T A \to \C \ent c_2 \T \D \leadsto c'_2
%   }{
%     \ctx \ent \letrecin{f \, x = c_1} c_2 \T \D \\ \leadsto \letrecin{f \, x = c'_1} c'_2 \T \D
%   }

%   \inferrule[Ret]{
%     \ctx \ent v \T A \leadsto v'
%   }{
%     \ctx \ent \ret v \T A \E \emptyset \leadsto \ret v' \T A \E \emptyrow
%   }

%   \inferrule[Op]{
%     (\op \T A \to B) \in \sig \\
%     \ctx \ent v \T A \leadsto v'
%   }{
%     \ctx \ent \op \, v \T B \E \{\op\} \leadsto \op \, v' \T B \E \{\op, .\}
%   }

%   \inferrule[Do]{
%     \ctx \ent c_1 \T \C \leadsto c'_1 \\
%     S = \forall \bar{\alpha} . A \\
%     \bar{\alpha} = FTV(A) - TV(\ctx) \\
%     \ctx, x \T S \ent c_2 \T \D \leadsto c'_2
%   }{
%     \begingroup\color{red}
%     \ctx \ent \doin{x \leftarrow c_1} c_2 \T \D \leadsto (\funtyped{x}{A}{c'_2})(\Lambda \bar{\alpha} . c'_1)
%     \endgroup
%   }

%   \inferrule[With]{
%     \ctx \ent v \T \C \hto \D \leadsto v' \\
%     \ctx \ent c \T \C \leadsto c'
%   }{
%     \ctx \ent \withhandle{v}{c} \T \D \leadsto \withhandle{v'}{c'} \T \D
%   }
% \end{mathpar}
% \end{minipage}
% }
% \end{center}
% \caption{Elaboration of source to core language: computations}\label{fig:elaboration:comp}
% \end{figure}

\subsection{Constraint Generation}
% Figure~\ref{fig:constraints:exp} and Figure~\ref{fig:constraints:comp} show the constraint generation algorithm for \core.

% \begin{figure}[!htb]
% \begin{center}
% \framebox{
% \begin{minipage}{0.95\columnwidth}
% \[\begin{array}{r@{~}c@{~}l}
%   \text{typing contexts}~\ctx & \bnfis {} & \epsilon \bnfor \ctx, x : A, x : \polytype{B}\\
% \end{array}\]
% \textbf{Expressions}
% \begin{mathpar}
%   \inferrule[Val]{
%     \ctx \ent v \T A \leadsto v'\\
%     A \le B
%   }{
%     \ctx \ent v \T B \leadsto v'
%   }

%   \inferrule[Var]{
%     (x \T S) \in \ctx \\
%     S = \forall \bar{\alpha} . A
%   }{
%     \ctx \ent x \T A[\bar{S}/\bar{\alpha}] \leadsto x \, \bar{S}
%   }

%   \inferrule[Const]{
%     (\const \T A) \in \sig
%   }{
%     \ctx \ent \const \T A \leadsto \const'
%   }

%   \inferrule[Fun]{
%     \ctx, x \T A \ent c \T \C \leadsto c'
%   }{
%     \ctx \ent \fun{x} c \T A \to \C \leadsto \funtyped{x}{A}{c'} \T A \to \C
%   }

%   \inferrule[Hand]{
%     \ctx, x \T A \ent c_r \T B \E \dirt \\
%     \Big[
%       (\op \T A_\op \to B_\op) \in \sig \qquad \\
%       \ctx, x \T A_\op, k \T B_\op \to B \E \dirt \ent c_\op \T B \E \dirt
%     \Big]_{\op \in \ops}
%   }{
%     \ctx \ent \shorthand \T \\ A \E \dirt \cup \ops \hto B \E \dirt \\ \leadsto \shorthand \T \\ A \E \dirt \cup \ops \hto B \E \dirt
%   }

% \end{mathpar}
% \end{minipage}
% }
% \end{center}
% \caption{Constraint generation within expressions}\label{fig:constraints:exp}
% \end{figure}


% \begin{figure}[!htb]
% \begin{center}
% \framebox{
% \begin{minipage}{0.95\columnwidth}
% \[\begin{array}{r@{~}c@{~}l}
%   \text{typing contexts}~\ctx & \bnfis {} & \epsilon \bnfor \ctx, x : A, x : \polytype{B}\\
% \end{array}\]
% \textbf{Computations}
% \begin{mathpar}
%   \inferrule[Comp]{
%     \ctx \ent c \T \C \leadsto c'\\
%     \C \le \D
%   }{
%     \ctx \ent c \T \D \leadsto c'
%   }

%   \inferrule[App]{
%     \ctx \ent v_1 \T A \to \C \leadsto v'_1 \\
%     \ctx \ent v_2 \T A \leadsto v'_2
%   }{
%     \ctx \ent v_1 \, v_2 \T \C \leadsto v'_1 \, v'_2 \T \C
%   }

%  \inferrule[LetRec]{
%     \ctx, f \T A \to \C, x \T A \ent c_1 \T \C \leadsto c'_1 s\\
%     \ctx, f \T A \to \C \ent c_2 \T \D \leadsto c'_2
%   }{
%     \ctx \ent \letrecin{f \, x = c_1} c_2 \T \D \\ \leadsto \letrecin{f \, x = c'_1} c'_2 \T \D
%   }

%   \inferrule[Ret]{
%     \ctx \ent v \T A \leadsto v'
%   }{
%     \ctx \ent \ret v \T A \E \emptyset \leadsto \ret v' \T A \E \emptyrow
%   }

%   \inferrule[Op]{
%     (\op \T A \to B) \in \sig \\
%     \ctx \ent v \T A \leadsto v'
%   }{
%     \ctx \ent \op \, v \T B \E \{\op\} \leadsto \op \, v' \T B \E \{\op, .\}
%   }

%   \inferrule[Do]{
%     \ctx \ent c_1 \T \C \leadsto c'_1 \\
%     S = \forall \bar{\alpha} . A \\
%     \bar{\alpha} = FTV(A) - TV(\ctx) \\
%     \ctx, x \T S \ent c_2 \T \D \leadsto c'_2
%   }{
%     \begingroup\color{red}
%     \ctx \ent \doin{x \leftarrow c_1} c_2 \T \D \leadsto (\funtyped{x}{A}{c'_2})(\Lambda \bar{\alpha} . c'_1)
%     \endgroup
%   }

%   \inferrule[With]{
%     \ctx \ent v \T \C \hto \D \leadsto v' \\
%     \ctx \ent c \T \C \leadsto c'
%   }{
%     \ctx \ent \withhandle{v}{c} \T \D \leadsto \withhandle{v'}{c'} \T \D
%   }
% \end{mathpar}
% \end{minipage}
% }
% \end{center}
% \caption{Constraint generation within computations}\label{fig:constraints:comp}
% \end{figure}

This chapter describes the type inference algorithm for \core. Type inference has been brievely explained in Chapter~\ref{type-inference-explain}. It is the process where types are automatically inferred by the compiler. Typically, a type inference algorithm uses contraint-based type inference rules. For \core, we cannot immediately jump into these rules. There are quite a few steps to go through first. 

Typically, a type inference algortihm uses unification to solve the constraints generated by the type inference algorithm. In the case of a Hindley-Milner system, we can solve all rules generated by the type inference algorithm. With subtyping, not all rules are solved and a type may still contain constraints after inference has been completed. Algebraic subtyping uses an analogue for unification, called biunification, to encode the constraints into the intersection and union types.

In order to use biunification, we require polar types. The idea of polar types is to distinguish between input types and output types. Input types are used to describe inputs, while output types are used to describe outputs. Input and output types are represented as, respectively, negative and positive types in the terms of polar types. 

If we ever have a program in which we have to choose to produce an output of type $\C_1$ or $\C_2$, the actual output type is $\C_1 \union \C_2$. This could happen in the case of an \texttt{match} or \texttt{if} statement. In other situations, we may be in a situation where an input is used in a case where type $\C_1$ is required and in a case where type $\C_2$ is required, then the input is given $\C_1 \intersection \C_2$ as its type. Equivalently, this can also happen with pure types $A_1$ and $A_2$.  

In other words, input types only use intersections and output types only use unions. Polar types do not allow this convention to be broken. Making this restriction simplifies the problem of solving subtyping constraints greatly and allow them to be solved. \cite{dolan2017algebraic, pottier1998type} 

\section{Polar types}\label{polarity}

Figure~\ref{fig:types:core:polar} shows the polar type of \core. The separation of the union $\union$ and the intersection $\intersection$ types can be clearly seen. With the exception of the handler type, all types are equivalent to the system of algebraic subtyping. 

\todo{explain recursive types?}

Polarity also extends to typing schemes as seen in the typing context $\ctxp$. A polar typing scheme $[\ctxm^-]A^+$ has a positive type $A^+$ and a monomorphic environment consisting of $\lambda$-bound variables with negative types. 

The inference algorithm works only with polar typing schemes. The reader may wonder whether polar typing schemes are enough to infer a principal typing scheme. Dolan has shown that polar typing schemes do suffice, I extend his proof to incorporate algebraic effects and handlers in Chapter~\ref{principality}. \cite{dolan2017algebraic}

\begin{figure}[!htb]
\begin{center}
\framebox{
\begin{minipage}{0.98\columnwidth}
\[\begin{array}{r@{~}c@{~}l@{\quad}l}
    \text{polymorphic typing contexts}~\ctxp & \bnfis {} & \epsilon \bnfor \ctxp, \letvar : [\ctxm^-]A^+ \\
    \text{(pure) type}~A^+, B^+ & \bnfis {}
    & \boolty & \text{bool type} \\
    & \bnfor & A^- \to \C^+ & \text{function type} \\
    & \bnfor & \C^- \hto \D^+ & \textbf{handler type} \\
    & \bnfor & \alpha & \text{type variable} \\
    & \bnfor & \rectype{A^+} & \text{recursive type} \\
    & \bnfor & \bot & \text{bottom} \\
    & \bnfor & A^+ \union B^+ & \text{union} \\
    \text{dirty type}~\C^+, \D^+ & \bnfis {} & A^+ \E \dirt^+ \\

    \text{(pure) type}~A^-, B^- & \bnfis {}
    & \boolty & \text{bool type} \\
    & \bnfor & A^+ \to \C^- & \text{function type} \\
    & \bnfor & \C^+ \hto \D^- & \textbf{handler type} \\
    & \bnfor & \alpha & \text{type variable} \\
    & \bnfor & \rectype{A^-} & \text{recursive type} \\
    & \bnfor & \top & \text{top} \\
    & \bnfor & A^- \intersection B^- & \text{intersection} \\
    \text{dirty type}~\C^-, \D^- & \bnfis {} & A^- \E \dirt^- \\

    \text{dirt}~\dirt^+ & \bnfis {} & \op & \text{operation} \\
    & \bnfor & \delta & \text{dirt variable} \\
    & \bnfor & \emptyrow & \text{empty dirt} \\
    & \bnfor & \dirt_1^+ \union \dirt_2^+ & \text{union} \\

    \text{dirt}~\dirt^- & \bnfis {} & \op & \text{operation} \\
    & \bnfor & \delta & \text{dirt variable} \\
    & \bnfor & \allops & \text{full dirt (all operations, top)} \\
    & \bnfor & \dirt_1^- \intersection \dirt_2^- & \text{intersection}
\end{array}\]
\end{minipage}
}
\end{center}
\caption{Polar types of \core}\label{fig:types:core:polar}
\end{figure}
    
\subsection{Unification}

To operate on polar type terms, we generalise from substitutions to bisubsti- tutions, which map type variables to a pair of a positive and a negative type term. The definitions for bisubstitions are given in Figure~\ref{fig:bisubstitution}.

\begin{figure}[!htb]
    \begin{center}
    \begin{framed}
    \begin{minipage}[t]{0.95\columnwidth}
    \begin{mathpar}    
        \inferrule[Bisubstitution]{}{
          \xi = [A^+ / \alpha^+, A^- / \alpha^-, \dirt^+/ \delta^+, \dirt^- / \delta^-]
        }\\

        \inferrule[]{}{
            \xi'(\alpha^+) = \alpha \\
            \xi'(\alpha^-) = \alpha \\
            \xi'(\delta^+) = \delta \\
            \xi'(\delta^-) = \delta \\
            \xi'(\_) = \_
        }

    \end{mathpar}
    \end{minipage}

    \begin{minipage}[t]{0.475\columnwidth}
    \begin{mathpar}
        \inferrule[]{}{
            \xi(\C^+) \equiv \xi(A^+ \E \dirt^+) \equiv \xi(A^+) \E \xi(\dirt^+)
        }\\

        \inferrule[]{}{
            \xi(\dirt_1^+ \union \dirt_2^+) \equiv \xi(\dirt_1^+) \union \xi(\dirt_2^+)
        }\\

        \inferrule[]{}{
            \xi(\op) \equiv \op
        }\\

        \inferrule[]{}{
            \xi(\emptyrow) \equiv \emptyrow
        }\\

        \inferrule[]{}{
            \xi(A_1^+ \union A_2^+) \equiv \xi(A_1^+) \union \xi(A_2^+)
        }\\
        
        \inferrule[]{}{
            \xi(\bot) \equiv \bot
        }\\
    
        \inferrule[]{}{
            \xi(bool) \equiv bool
        }\\
    
        \inferrule[]{}{
            \xi(A^- \to A^+) \equiv \xi(A^-) \to \xi(A^+)
        }\\
    
        \inferrule[]{}{
            \xi(A^- \hto A^+) \equiv \xi(A^-) \hto \xi(A^+)
        }\\
    
        \inferrule[]{}{
            \xi(\rectype{A^+}) \equiv \rectype{\xi'(A^+)}
        }
    \end{mathpar}
    \end{minipage}
    \begin{minipage}[t]{0.475\columnwidth}
    \begin{mathpar}
        \inferrule[]{}{
            \xi(\C^-) \equiv \xi(A^- \E \dirt^-) \equiv \xi(A^-) \E \xi(\dirt^-)
        }\\

        \inferrule[]{}{
            \xi(\dirt_1^- \intersection \dirt_2^-) \equiv \xi(\dirt_1^-) \intersection \xi(\dirt_2^-)
        }\\

        \inferrule[]{}{
            \xi(\dirt_1^- \union \dirt_2^-) \equiv \xi(\dirt_1^-) \union \xi(\dirt_2^-)
        }\\

        \inferrule[]{}{
            \xi(\op) \equiv \op
        }\\

        \inferrule[]{}{
            \xi(\allops) \equiv \allops
        }\\

        \inferrule[]{}{
            \xi(A_1^- \intersection A_2^-) \equiv \xi(A_1^-) \intersection \xi(A_2^-)
        }\\
    
        \inferrule[]{}{
            \xi(\top) \equiv \top
        }\\
    
        \inferrule[]{}{
            \xi(bool) \equiv bool
        }\\
    
        \inferrule[]{}{
            \xi(A^+ \to A^-) \equiv \xi(A^+) \to \xi(A^-)
        }\\
    
        \inferrule[]{}{
            \xi(A^+ \hto A^-) \equiv \xi(A^+) \hto \xi(A^-)
        }\\
    
        \inferrule[]{}{
            \xi(\rectype{A^-}) \equiv \rectype{\xi'(A^-)}
        }
    \end{mathpar}
    \end{minipage}

    \end{framed}
    \end{center}
\caption{Bisubstitutions}\label{fig:bisubstitution}
\end{figure}

The presence of explicit type application in F, F$_\omega$ and CoC makes the exact parameterisation of a polymorphic type relevant. Conversely, in ML, the parameterisation is irrelevant and all that matters is the set of possible instances.

\begin{figure}[!htb]
\begin{center}
\begin{framed}
\begin{minipage}[t]{0.95\columnwidth}
\begin{mathpar}    
    \inferrule[]{}{
        \forall \alpha \forall \beta . \alpha \to \beta \to \alpha \\
        \forall \beta \forall \alpha . \alpha \to \beta \to \alpha
    }\\

    \inferrule[]{}{
        \{ \alpha \to \beta \to \alpha \bnfor{} \alpha, \beta \text{types}\} \\ 
        \{ \alpha \to \beta \to \alpha \bnfor{} \beta, \alpha \text{types}\}
    }
\end{mathpar}
\end{minipage}
\end{framed}
\end{center}
\caption{Parameterisation and typing}\label{fig:parameterisation}
\end{figure}

Thus, when manipulating constraints, an ML type checker need only preserve equivalence of the set of instances, and not equivalence of the parameterisation. This freedom is not much used in plain ML, since unification happens to preserve equivalence of the parameterisation. However, this freedom is what allows MLsub to eliminate subtyping constraints.

For all positive type terms $A+$ and variables, there exist positive type terms $A^+_\alpha$ and $A^+_g$ such that $A^+_\alpha \in {\bot, \alpha}$, $\alpha$ is guarded in $A^+_g$, and $A^+$ is equivalent to $A^+_\alpha \union A^+_g$.

For all negative type terms $A-$ and variables, there exist negative type terms $A^-_\alpha$ and $A^-_g$ such that $A^-_\alpha \in {\top, \alpha}$, $\alpha$ is guarded in $A^-_g$, and $A^-$ is equivalent to $A^-_\alpha \intersection A^-_g$.

\begin{figure}[!htb]
\begin{center}
\begin{framed}
\begin{minipage}[t]{0.95\columnwidth}
\begin{mathpar} 
    \inferrule[]{}{
        \rectypep{A^+} = \rectype{A^+_g} \\
        \rectypen{A^-} = \rectype{A^-_g}
    }
    
\end{mathpar}
\end{minipage}
\end{framed}
\end{center}
\caption{Polar recursive type}\label{fig:recursive}
\end{figure}

\begin{figure}[!htb]
\begin{center}
\begin{framed}
\begin{minipage}[t]{0.95\columnwidth}
\textbf{Constructed (predicate): \\constructed($A$)}
\begin{mathpar} 
    \inferrule[]{}{
        constructed(A \to \C)
    }\\
    \inferrule[]{}{
        constructed(\C \hto \D)
    }\\
    \inferrule[]{}{
        constructed(bool)
    }
\end{mathpar}
\end{minipage}
\end{framed}
\end{center}
\caption{Constructed types}\label{fig:constructed}
\end{figure}

\begin{figure}[!htb]
\begin{center}
\begin{framed}
\begin{minipage}[t]{0.95\columnwidth}
\textbf{Atomic (partial function): \\atomic($A^+ \le A^-$) = $\theta$, atomic($\dirt^+ \le \dirt^-$) = $\theta$}
\begin{mathpar} 
    \inferrule[]{
        constructed(A^-) \\
        \beta \text{ not free in } A^-
    }{
        atomic(\beta \le A^-) = [\beta \intersection A^- / \beta^-]
    }\\
    \inferrule[]{
        constructed(A^-) \\
        \beta \text{ free in } A^-
    }{
        atomic(\beta \le A^-) = [\rectypen{(\beta \intersection [\alpha / \beta^-](A^-))} / \beta^-]
    }\\
    \inferrule[]{
        constructed(A^+) \\
        \beta \text{ not free in } A^-
    }{
        atomic(A^+ \le \beta) = [\beta \union A^+ / \beta^+]
    }\\
    \inferrule[]{
        constructed(A^+) \\
        \beta \text{ free in } A^-
    }{
        atomic(A^+ \le \beta) = [\rectypep{(\beta \union [\alpha / \beta^+](A^+))} / \beta^+]
    }\\
    \inferrule[]{}{
        atomic(\beta \le \gamma) = [\rectypen{(\beta \intersection [\alpha / \beta^-](\gamma))} / \beta^-] \equiv [\beta \intersection \gamma/\beta^-] \equiv [\beta \union \gamma/\gamma^+]
    }\\
    
    \inferrule[]{}{
        atomic(\delta \le \op^-) = [\delta \intersection \op / \delta^-]
    }\\
    \inferrule[]{}{
        atomic(\op^+ \le \delta) = [\delta \union \op / \delta^+]
    }\\
    \inferrule[]{}{
        atomic(\delta \le (\dirt_1 \union \dirt_2)) = [\delta \intersection (\dirt_1 \union \dirt_2) / \delta^-]
    }\\
    \inferrule[]{}{
        atomic(\delta_1 \le \delta_2) = [\delta_1 \union \delta_2 / \delta_2^+] \equiv [\delta_1 \intersection \delta_2 / \delta_1^-]
    }
\end{mathpar}
\end{minipage}
\end{framed}
\end{center}
\caption{Constraint solving}\label{fig:constraints}
\end{figure}

\begin{figure}[!htb]
\begin{center}
\begin{framed}
\begin{minipage}[t]{0.95\columnwidth}
\textbf{Subi (partial function): \\subi($A^+ \le A^-$) = $C$, subi($\dirt^+ \le \dirt^-$) = $C$, subi($\C^+ \le \C^-$) = $C$}
\begin{mathpar}    
    \inferrule[]{}{
        subi(A^+ \E \dirt^+ \le A^- \E \dirt^-) = \{A^+ \le A^-, \dirt^+ \le \dirt^-\}
    }\\
    \inferrule[]{}{
        subi(A^-_1 \to \C^+_1 \le A^+_2 \to \C^-_2) = \{A^+_2 \le A^-_1, \C^+_1\le \C^-_2\}
    }\\
    \inferrule[]{}{
        subi(\C^-_1 \hto \D^+_1 \le \C^+_2 \hto \D^-_2) = \{\C^+_2 \le \C^-_1, \D^+_1\le \D^-_2\}
    }\\
    \inferrule[]{}{
        subi(bool \le bool) = \{\}
    }\\
    \inferrule[]{}{
        subi(\rectype{A^+} \le A-) = \{[\rectype{A^+} / \alpha](A^+) \le A^-\}
    }\\
    \inferrule[]{}{
        subi(A^+ \le \rectype{A^-}) = \{A^+ \le [\rectype{A^-} / \alpha]A^- \}
    }\\
    \inferrule[]{}{
        subi(A^+_1 \union A^+_2 \le A^-) = \{A^+_1 \le A^-, A^+_2 \le A^-\}
    }\\
    \inferrule[]{}{
        subi(A^+ \le A^-_1 \intersection A^-_2) = \{A^+ \le A^-_1, A^+ \le A^-_2\}
    }\\
    \inferrule[]{}{
        subi(\bot \le A^-) = \{\}
    }\\
    \inferrule[]{}{
        subi(A^+ \le \top) = \{\}
    }\\
    \inferrule[]{}{
        subi(\op \le \op) = \{\}
    }\\
    \inferrule[]{}{
        subi(\dirt^+_1 \union \dirt^+_2 \le \dirt^-) = \{\dirt^+_1 \le \dirt^-, \dirt^+_2 \le \dirt^-\}
    }\\
    \inferrule[]{}{
        subi(\dirt^+ \le \op \union \dirt^-) = \{\dirt^+ \le \dirt^-\}
    }\\
    \inferrule[]{}{
        subi(\dirt^+ \le \dirt^- \union \op) = \{\dirt^+ \le \dirt^-\}
    }\\
    \inferrule[]{}{
        subi(\dirt^+ \le \dirt^-_1 \union \dirt^-_2) = \{\dirt^+ \le \dirt^-_1, \dirt^+ \le \dirt^-_2\}
    }\\
    \inferrule[]{}{
        subi(\dirt^+ \le \dirt^-_1 \intersection \dirt^-_2) = \{\dirt^+ \le \dirt^-_1, \dirt^+ \le \dirt^-_2\}
    }\\
    \inferrule[]{}{
        subi(\emptyrow \le \dirt^-) = \{\}
    }\\
    \inferrule[]{}{
        subi(\dirt^+ \le \allops) = \{\}
    }
\end{mathpar}
\end{minipage}
\end{framed}
\end{center}
\caption{Constraint decomposition}\label{fig:constraints-decompose}
\end{figure}

\begin{figure}[!htb]
\begin{center}
\begin{framed}
\begin{minipage}[t]{0.95\columnwidth}
\textbf{Binunify(History, ContraintSet) = substitution}
\begin{mathpar}
    \inferrule[Start]{}{
        \mathit{biunify}(C) = \mathit{biunify}(\emptyset; C)
    }\\
    \inferrule[Empty]{}{
        \mathit{biunify}(H; \epsilon) = 1
    }\\
    \inferrule[Redundant]{
        c \in H
    }{
        \mathit{biunify}(H; c :: C) = \mathit{biunify}(H; C)
    }\\
    \inferrule[Atomic]{
        atomic(c) = \theta_c
    }{
        \mathit{biunify}(H; c :: C) = \mathit{biunify}(\theta_c(H \cup \{c\}); \theta_c(C)) \cdot \theta_c
    }\\
    \inferrule[Decompose]{
        subi(c) = C'
    }{
        \mathit{biunify}(H; c :: C) = \mathit{biunify}(H \cup \{c\}; C' \concat C)
    }
\end{mathpar}
\end{minipage}
\end{framed}
\end{center}
\caption{Biunification algorithm}\label{fig:biunification}
\end{figure}
\subsection{Principal Type Inference}
\subsection{Deciding Subsumption}

\section{Proofs}
\subsection{Instantiation}


\subsection{Weakening}


\subsection{Substitution}


\subsection{Soundness}

\section{Implementation}
Describe the approach itself, in such detail that a reader could also implement this approach if s/he wished to do that.

\section{Evaluation}
Novel approaches to problems are often evaluated empirically. Describe the evaluation process in such detail that a reader could reproduce the results. Describe in detail the setup of an experiment. Argue why this experiment is useful, and what you could learn from it. Be precise about what you want to measure, or about the hypothesis that you are testing. Discuss and interpret the results in terms of your experimental questions. Summarize the conclusions of the experimental evaluation.

\section{Conclusion}
Briefly recall what the goal of the work was. Summarize what you have done, summarize the results, and present conclusions. Conclusions include a critical assessment: where the original goals reached? Discuss the limitations of your work. Describe how the work could possibly be extended in the future, mitigating limitations or solving remaining problems.

\appendixpage*          % if wanted
\appendix
% \chapter{Appendix A}
% \input{tex/appendix}

\backmatter

% The bibliography comes after the appendices.
% You can replace the standard "abbrv" bibliography style by another one.
\bibliographystyle{abbrv}
\bibliography{bib/main}

\end{document}
