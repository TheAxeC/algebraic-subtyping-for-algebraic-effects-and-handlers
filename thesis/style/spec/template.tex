\documentclass[sigplan,10pt]{acmart}\settopmatter{printfolios=true}
% \documentclass[acmsmall,10pt]{acmart}\settopmatter{printfolios=true}
%% For final camera-ready submission
%\documentclass[sigplan,10pt]{acmart}\settopmatter{}

\settopmatter{printacmref=false} % Removes citation information below abstract
\renewcommand\footnotetextcopyrightpermission[1]{} % removes footnote with conference information in first column
\pagestyle{plain} % removes running headers

%% Some recommended packages.
\usepackage{booktabs}   %% For formal tables:
                        %% http://ctan.org/pkg/booktabs
\usepackage{subcaption} %% For complex figures with subfigures/subcaptions
                        %% http://ctan.org/pkg/subcaption

\usepackage{mathpartir}
\usepackage{xspace}
\usepackage{stmaryrd}
\usepackage{listings}
\usepackage{newtxmath}

%% Tikz Needed packages
\usepackage{pgfplots}
\pgfplotsset{width=7cm,compat=1.8}
\usepackage{pgfplotstable}
\renewcommand*{\familydefault}{\sfdefault}

%\makeatletter\if@ACM@journal\makeatother
%%% Journal information (used by PACMPL format)
%%% Supplied to authors by publisher for camera-ready submission
%\acmJournal{ICFP 17 student research competition}
%\acmVolume{1}
%\acmNumber{1}
%\acmArticle{1}
%\acmYear{2017}
%\acmMonth{1}
%\acmDOI{10.1145/nnnnnnn.nnnnnnn}
%\startPage{1}
%\else\makeatother
%%% Conference information (used by SIGPLAN proceedings format)
%%% Supplied to authors by publisher for camera-ready submission
%\acmConference[ICFP 2017 Student Research Competition]{ICFP 2017 Student Research Competition}{September, 2017}{Oxford, UK}
%\acmYear{2017}
%\acmISBN{978-x-xxxx-xxxx-x/YY/MM}
%\acmDOI{10.1145/nnnnnnn.nnnnnnn}
%\startPage{1}
%\fi

%% Copyright information
%% Supplied to authors (based on authors' rights management selection;
%% see authors.acm.org) by publisher for camera-ready submission
\setcopyright{none}             %% For review submission
%\setcopyright{acmcopyright}
%\setcopyright{acmlicensed}
%\setcopyright{rightsretained}
%\copyrightyear{2017}           %% If different from \acmYear

%% Bibliography style
\bibliographystyle{ACM-Reference-Format}
%% Citation style
%% Note: author/year citations are required for papers published as an
%% issue of PACMPL.
%\citestyle{acmauthoryear}  %% For author/year citations
%\citestyle{acmnumeric}     %% For numeric citations
%\setcitestyle{nosort}      %% With 'acmnumeric', to disable automatic
                            %% sorting of references within a single citation;
                            %% e.g., \cite{Smith99,Carpenter05,Baker12}
                            %% rendered as [14,5,2] rather than [2,5,14].
%\setcitesyle{nocompress}   %% With 'acmnumeric', to disable automatic
                            %% compression of sequential references within a
                            %% single citation;
                            %% e.g., \cite{Baker12,Baker14,Baker16}
                            %% rendered as [2,3,4] rather than [2-4].

$for(include)$
\input{$include$}
$endfor$

\newcommand{\todo}[1]{\textcolor{red}{\textsc{Todo:} #1}}

\begin{document}

\title{Algebraic subtyping for algebraic effects and handlers}

\author{Axel Faes}
\affiliation{
  \position{student : undergraduate\\ACM Student Member: 2461936}
  \department{Department of Computer Science}
  \institution{KU Leuven}
}
\email{axel.faes@student.kuleuven.be}

\author{Amr Hany Saleh}
\affiliation{
  \position{daily advisor}
  \department{Department of Computer Science}
  \institution{KU Leuven}
}
\email{amrhanyshehata.saleh@kuleuven.be }

\author{Tom Schrijvers}
\affiliation{
  \position{promotor}
  \department{Department of Computer Science}
  \institution{KU Leuven}
}
\email{tom.schrijvers@kuleuven.be}

%% Keywords
%% comma separated list
\keywords{algebraic effect handler, algebraic subtyping, effects, optimised compilation}  %% \keywords is optional

%% Abstract
%% Note: \begin{abstract}...\end{abstract} environment must come
%% before \maketitle command
\begin{abstract}
Algebraic effects and handlers are a very active area of research. An important aspect is the development of an optimising compiler. \eff is an ML-style language with support for effects and forms the testbed for the optimising compiler. However, the type-\&-effect system of \eff is unsatisfactory. This is due to the lack of some elegant properties. It is also awkward to implement and use in practice.
\end{abstract}

\maketitle

$body$

%% Acknowledgments
\begin{acks}
  I would like to thank Amr Hany Saleh for his continuous guidance and help.
\end{acks}

\bibliographystyle{ACM-Reference-Format}
\bibliography{$bibliography$}

\end{document}
