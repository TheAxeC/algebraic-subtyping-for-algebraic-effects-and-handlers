\documentclass[master=cws,masteroption=ai]{kulemt}
\setup{title={Algebraic Subtyping for Algebraic Effects and Handlers},
  author={Axel Faes},
  promotor={Prof.\,dr.\,ir.\ Tom Schrijvers},
  assessor={Ir.\,Kn. Owsmuch\and K. Nowsrest},
  assistant={Amr Hany Saleh}}
% The following \setup may be removed entirely if no filing card is wanted
\setup{filingcard,
  translatedtitle=,
  udc=,
  shortabstract={Here comes a very short abstract, containing no more than 500
    words. \LaTeX\ commands can be used here. Blank lines (or the command
    \texttt{\string\pa r}) are not allowed!
    \endgraf}}
% Uncomment the next line for generating the cover page
%\setup{coverpageonly}
% Uncomment the next \setup to generate only the first pages (e.g., if you
% are a Word user.
%\setup{frontpagesonly}

% Choose the main text font (e.g., Latin Modern)
\setup{font=lm}

% If you want to include other LaTeX packages, do it here.

% Finally the hyperref package is used for pdf files.
% This can be commented out for printed versions.
\usepackage[pdfusetitle,colorlinks,plainpages=false]{hyperref}

%\includeonly{chap-n}
\begin{document}

\begin{preface}
  I would like to thank everybody who kept me busy the last year,
  especially my promoter and my assistants. I would also like to thank the
  jury for reading the text. My sincere gratitude also goes to my wive and
  the rest of my family.
\end{preface}

\tableofcontents*

\begin{abstract}
  The \texttt{abstract} environment contains a more extensive overview of
  the work. But it should be limited to one page.
\end{abstract}

% A list of figures and tables is optional
%\listoffigures
%\listoftables
% If you only have a few figures and tables you can use the following instead
\listoffiguresandtables
% The list of symbols is also optional.
% This list must be created manually, e.g., as follows:
\chapter{List of Abbreviations and Symbols}
\section*{Abbreviations}
\begin{flushleft}
  \renewcommand{\arraystretch}{1.1}
  \begin{tabularx}{\textwidth}{@{}p{12mm}X@{}}
    LoG   & Laplacian-of-Gaussian \\
    MSE   & Mean Square error \\
    PSNR  & Peak Signal-to-Noise ratio \\
  \end{tabularx}
\end{flushleft}
\section*{Symbols}


% Now comes the main text
\mainmatter

% \include{intro}
% \include{chap-1}
% \include{chap-2}
% % ... and so on until
% \include{chap-n}
% Algebraic effects and handlers are a very active area of research. An important aspect is the development of an optimising compiler. Without a type-\&-effect system with explicit typing, it is easy for type checking bugs to be introduced during the construction of optimised compilation. A core language with row-based effects was introduced. The core language is explicitly typed in order to reduce bugs in the optimised compilation.

%
% % If you have appendices:
% \appendixpage*          % if wanted
% \appendix
% \include{app-A}
% % ... and so on until
% \include{app-n}
$body$

\backmatter

% The bibliography comes after the appendices.
% You can replace the standard "abbrv" bibliography style by another one.
\bibliographystyle{abbrv}
\bibliography{references}

\end{document}

%%% Local Variables:
%%% mode: latex
%%% TeX-master: t
%%% End:
