\section{Principal Type Inference}\label{principality}

We introduce a judgement form $\ctxp \prinent v : [\ctxm^-]A^+$ , stating that $[\ctxm^-]A^+$ is the principal typing scheme of $v$ under the polar typing context $\ctxp$. Similarly, $\ctxp \prinent c : [\ctxm^-]\C^+$ states that $[\ctxm^-]\C^+$ is the principal typing scheme of $v$ under the polar typing context $\ctxp$.

Principality for most syntactic elements are equivalent to the principality from algebraic subtyping \cite{dolan2017algebraic}. The novelty in presented in Chapter~\ref{novel-infer}, this is the principality of the algebraic effects and handler related syntactic elements. The other cases for principality are included in order to provide the complete type inference algorithm. 

The following subsections discuss the principality for all different constructs. The final type inference rules can be found in Figure~\ref{fig:inference:expressions} for expressions and Figure~\ref{fig:inference:computations} for computations.

\subsection{Principality for Functions}
We begin with the principality of $\lambda$-bound variables, functions and applications. In Figure~\ref{fig:inference:expressions} and Figure~\ref{fig:inference:computations}, these are represented by the inference rules \textsc{Var-$\ctxm$}, \textsc{Fun} and \textsc{App}.

\paragraph{$\lambda$-bound variables} $\lambda$-bound variables $x$ are typed with the \textsc{Var-$\ctxm$} from Figure~\ref{fig:core-reform-typing}. in this rule, a $\lambda$-bound variable $x$ can be given the typing scheme $[x : A]A$. The principal typing scheme is the scheme that subsumes any typing scheme of the form $[x : A]A$, this is the typing scheme $[x : \alpha]\alpha$.

\paragraph{Functions} Functions $\fun{x}c$ are typed with the \textsc{Fun} rule from Figure~\ref{fig:core-reform-typing}. This requires that $c$ can be typed. Otherwise, if $c$ cannot be typed, then $\fun{x}c$ cannot be typed. In that case, there is no principal type. If $c$ can be typed, it has the principal polar typing scheme $[\ctxm^-]\C^+$.

The general form of a typing derivation for $\fun{x}c$ is:

$\inferrule*[left=Fun]{
    \inferrule*[left=SubComp]{
        \ctxp \entt c \T [\ctxm_1^-]\C^+
      }{
        \ctxp \entt c \T [\ctxm_2, x:A]\C
      }
}{
    \ctxp \entt \fun{x}c \T [\ctxm_2](A \to \C)
}$

\textsc{SubComp} can only be applied if $[\ctxm_1^-]\C^+ \le^\forall [\ctxm_2, x:A]\C$. We can write this as: $[\ctxm_1^-\setminus x, x : \ctxm_1^-(x)]\C^+ \le^\forall [\ctxm_2, x:A]\C$. By function subtyping, this has to be equivalent to: $[\ctxm_1^-\setminus x](\ctxm_1^-(x) \to \C^+) \le^\forall [\ctxm_2](A \to \C)$. From this, we can conclude that $[\ctxm_1^-\setminus x](\ctxm_1^-(x) \to \C^+)$ is the principal type of a function.

\paragraph{Applications} Applications $v_1 \, v_2$ are typed with the \textsc{App} rule from Figure~\ref{fig:core-reform-typing}. Both $v_1$ and $v_2$ need to be typable in order to be able to type $v_1 \, v_2$. Let the principal types of $v_1$ and $v_2$ be $[\ctxm_1^-]A^+_1$ and $[\ctxm_2^-]A^+_2$.

The general form of a typing derivation for $v_1 \, v_2$ is:

$\inferrule*[left=Fun]{
    \inferrule*[left=SubVal]{
        \ctxp \entt v_1 \T [\ctxm_1^-]A^+_1
      }{
        \ctxp \entt v_1 \T [\ctxm](B \to \D)
      }\\
      \inferrule*[left=SubVal]{
        \ctxp \entt v_2 \T [\ctxm_2^-]A^+_2
      }{
        \ctxp \entt v_2 \T [\ctxm]B
      }
}{
    \ctxp \entt v_1 \, v_2 \T [\ctxm]\D
}$

\textsc{SubVal} can only be applied if $[\ctxm_1^-]A^+_1 \le^\forall [\ctxm](B \to \D)$ and $[\ctxm_2^-]A^+_2 \le^\forall [\ctxm]B$. Since typing schemes are closed, $[\ctxm_1^-]A^+_1$ and $[\ctxm_2^-]A^+_2$ do not share typing variables. Therefor, we can use a (bi)substitution to find the most principal scheme. We get $\rho(A^+_1) \le B \to \D$, $\rho(A^+_2) \le B$, $\ctxm \le \rho(\ctxm_1^-)$ and $\ctxm \le \rho(\ctxm_2^-)$.

By introducing type variables $\alpha$ and $\beta$ and a dirt variable $\delta$ and taking $B = \rho(\alpha)$ and $\D = \rho(\beta \E \delta)$, we get other constraints $\rho(A^+_1) \le \rho(\alpha \to (\beta \E \delta))$, $\rho(A^+_2) \le \alpha$ and  $\rho([\ctxm_1^- \intersection \ctxm_2^-](\beta \E \delta)) \le [\ctxm]\D$. Any type that solves these constraints, types the application using the principal types for $v_1$ and $v_2$. The reason we see $[\ctxm_1^- \intersection \ctxm_2^-]$ is that that set of variables is the least set that is needed. 

In similar reason to the principality for $\lambda$-bound variables, we know that $\rho([\ctxm_1^- \intersection \ctxm_2^-](\beta \E \delta)$ is the most principal type for the application. The bisubstitution $\rho$ does need to be chosen correctly for this. The correct choice is any bisubstitution $\xi$ that solves the constraints $\{A^+_1 \le (\alpha \to (\beta \E \delta)), A^+_2 \le \alpha\}$. 

The bisubstitution $\xi$ can be computed using the biunification algorithm. We can make one final optimization. We can use the following simplified constraint set $\{A^+_1 \le  (A^+_2 \to (\beta \E \delta))\}$. $\alpha$ is only used twice. When $A^+_2 \le \alpha$ is being solved, $A^+_2$ will be substituted into $A^+_1 \le (\alpha \to (\beta \E \delta))$, which becomes $A^+_1 \le  (A^+_2 \to (\beta \E \delta))$.

\subsection{Principality for Handlers}\label{novel-infer}
We continue with the principality of handlers, handling, return and operations. In Figure~\ref{fig:inference:expressions} and Figure~\ref{fig:inference:computations}, these are represented by the inference rules \textsc{Hand}, \textsc{With}, \textsc{Ret} and \textsc{Op}.

\paragraph{Handlers} A handler $\shorthand$ is typed with the \textsc{Hand} rule from Figure~\ref{fig:core-reform-typing}. A handler has a value case $\ret x \mapsto c_r$ and some effect cases $\call{\op}{y}{k} \mapsto c_\op$ for each operation that the handler handles. If $c_r$ or any of $c_\op$ isn't typeable, then the handler isn't typeable either. Otherwise, $c_r$ has principal type $[\ctxm^-_r](B^+ \E \dirt^+)$. Every $c_\op$ has principal type $[\ctxm^-_\op](\C^+_\op)$. 

The general form of a typing derivation for $\shorthand$ is:

$\inferrule*[left=Hand]{
    \inferrule*[left=SubComp]{
        \ctxp \entt c_r \T [\ctxm^-_r](B^+ \E \dirt^+)
      }{
        \ctxp \entt c_r \T [\ctxm, x \T A](B \E \dirt)
      }\\
    \Big[
      (\op \T A_\op \to B_\op) \in \sig \qquad
    %   \ctxp \entt c_\op \T [\ctxm, y \T A_\op, k \T B_\op \to B \E \dirt](B \E \dirt)
      \inferrule*[left=SubComp]{
        \ctxp \entt c_\op \T [\ctxm^-_\op](\C^+_\op)
      }{
        \ctxp \entt c_\op \T [\ctxm, y \T A_\op, k \T B_\op \to B \E \dirt](B \E \dirt)
      }
    \Big]_{\op \in \ops}
  }{
    \ctxp \entt \shorthand \T [\ctxm](A \E \dirt \intersection \ops \hto B \E \dirt)
  }$

%   \scriptsize
% \begin{mathpar}
% \inferrule* [left = \scriptsize \textsc{TyHand}]
%     { \inferrule*[left = \scriptsize \textsc{TyReturn}]
%                 {\{y:int\}\ent y:\intty }
%                 {\{y:int\}\ent \\\\ \ret y:\intty \E \emptyset}
%       \quad
%       \inferrule*[left = \scriptsize \textsc{TyApp}]
%                 {\{i:\intty,\; k:\intty \to \intty \E \emptyset\}  \ent k \T \intty \E \emptyset \\\\
%                  \{i:\intty,\; k:\intty \to \intty \E \emptyset\}  \ent (i+1) \T \intty}
%                 {\{i:\intty,\; k:\intty \to \intty \E \emptyset\} \ent \\\\ k\; (i+1) \T \intty \E \emptyset }
%     }
%     {\emptyset \ent h \T \intty \E {\{\textit{inc}\}} \hto \intty \E \emptyset}
% \end{mathpar}
% \normalsize

\textsc{SubComp} can only be applied if $[\ctxm^-_r](B^+ \E \dirt^+) \le^\forall [\ctxm, x \T A](B \E \dirt)$ and $[\ctxm^-_\op](\C^+_\op) \le^\forall [\ctxm, y \T A_\op, k \T B_\op \to B \E \dirt](B \E \dirt)$. We can write the first subtyping constraint as: $[\ctxm_r^-\setminus x, x : \ctxm_1^-(x)](B^+ \E \dirt^+) \le^\forall [\ctxm, x:A]\C$. By function subtyping, this has to be equivalent to: $[\ctxm_r^-\setminus x](\ctxm_r^-(x) \to (B^+ \E \dirt^+)) \le^\forall [\ctxm](A \to \C)$. From this, we can conclude that $[\ctxm_r^-\setminus x](\ctxm_r^-(x) \to (B^+ \E \dirt^+))$ is the principal type of the value case.

Going to the operation cases, the subtyping constraint is $[\ctxm^-_\op](\C^+_\op) \le^\forall [\ctxm, y \T A_\op, k \T B_\op \to B \E \dirt](B \E \dirt)$. By reasoning in the same way, we can get the following: $[\ctxm^-_\op \setminus y \setminus k](\ctxm_\op^-(y) \to \ctxm_\op^-(k) \to \C^+_\op) \le^\forall [\ctxm](A_\op \to (B_\op \to B \E \dirt) \to B \E \dirt)$. 

In order to know the principal type of our handler, a few more steps are required. One aspect that we ignored up till now is that, for the operation cases, the right-hand side refers to the type of the handler. $\ctxm$, $B \E \dirt$ and $A$ occur in $[\ctxm](A \E \dirt \intersection \ops \hto B \E \dirt)$. In order to find the type of the handler, we need to use a similar reasoning as for the application. When this is done, we find that $\xi([\ctxm^-_r \intersection (\bigsqcap \ctxm^-_\op | \op \in \ops)](\ctxm_r^-(x) \E \delta \intersection \ops \hto \alpha \E \delta))$ with $\xi$ solving the constraint set $\{B^+ \E \dirt^+ \le (\alpha \E \delta), [A^+_\op \le \ctxm_\op^-(y), (B^+_\op \to (\alpha \E \delta)) \le \ctxm_\op^-(k), \C^+_\op \le (\alpha \E \delta)]_{\op \in \ops}\}$ is the typing scheme of the handler.

We want the handler to explicitly contain the information that the input of the handler may contain the handled effects, hence the occurence of $\intersection \ops$. However, we can give a more formal explanation.

For this, we need to slightly change the typing rule \textsc{Hand} into:

$\inferrule*[left=Hand]{
    \Psi_{\dirt_1} \\
  \ctxp \entt c_r \T [\ctxm, x \T A](B \E \dirt_2) \\
  \Big[
    \Psi_{\op} \qquad
    (\op \T A_\op \to B_\op) \in \sig \qquad 
    \ctxp \entt c_\op \T [\ctxm, y \T A_\op, k \T B_\op \to B \E \dirt_2](B \E \dirt_2)
  \Big]_{\op \in \ops}
}{
  \ctxp \entt \shorthand \T [\ctxm](A \E \dirt_1 \hto B \E \dirt_2)
}$

$\Psi_{\dirt_1}$ is used to make sure that any operation in the input dirt $\dirt_1$ that is not handled, is also present in the output $\dirt_2$. We can define $\Psi_{\dirt_1}$ as $\dirt_1 \le \dirt_2$ since that subtyping relation encodes that necessary information. 

$\Psi_{\op}$ gives us information about the structure of the handler. When a handler can handle effect $\op$, we expect the handler type to convey that information. More specifically, we expect the handler type to be $A \E \op \hto B \E \dirt_2$.  When deriving the principal type, $\Psi_{\op}$ will turn into a constraint. After the type and dirt variables have been introduced, the constraint takes the form $\ctxm_r^-(x) \E \op \hto (\alpha \E \delta) \le \ctxm_r^-(x) \E \delta_1 \hto (\alpha \E \delta)$. 

This is still a rather complicated constraint which we can manually simplify. By handler subtyping, this constraint decomposes into $(\alpha \E \delta) \le (\alpha \E \delta)$ and $\ctxm_r^-(x) \E \delta_1 \le \ctxm_r^-(x) \E \op$. The first constraint is trivial and can be removed. The second constraint can be decomposed further into $\ctxm_r^-(x) \le \ctxm_r^-(x)$ and $\delta_1 \le \op$. Again, the first constraint is trivial. In the end, each $\Psi_{\op}$ creates $\delta_1 \le \op$. 

Both $\dirt_1 \le \dirt_2$ and $\delta_1 \le \op$ are atomic. We can handle them using the following constraint solving equations (from Figure~\ref{fig:constraints}).

$atomic(\delta_1 \le \delta_2) = [\delta_1 \union \delta_2 / \delta_2^+] \equiv [\delta_1 \intersection \delta_2 / \delta_1^-]$

$atomic(\delta \le \op) = [\delta \intersection \op / \delta^-]$

Solving these constraints gives us:

$\xi([\ctxm^-_r \intersection (\bigsqcap \ctxm^-_\op | \op \in \ops)](\ctxm_r^-(x) \E (\bm{\delta_2 \intersection (\bigsqcap \op | \op \in \ops)}) \hto \alpha \E \delta_2))$ with $\xi$ solving the constraint set $\{B^+ \E \dirt^+ \le (\alpha \E \delta_2), [A^+_\op \le \ctxm_\op^-(y), (B^+_\op \to (\alpha \E \delta_2)) \le \ctxm_\op^-(k), \C^+_\op \le (\alpha \E \delta_2)]_{\op \in \ops}\}$.

$\delta_2 \intersection (\bigsqcap \op | \op \in \ops)$ is the same as $\delta_2 \intersection \ops$. As an additional note, the constraint set allows the occurence of $\delta_2$ in the output (the positive polarity occurence) to grow to account for effects introduced in the effect clauses themselves.

\paragraph{Handling} A with $\withhandle{v}{c}$ is typed with the \textsc{With} rule from Figure~\ref{fig:core-reform-typing}. If any of $v$ and $c$ are untypeable, then so is the with. If they are all typeable, they respectively have principal types $[\ctxm^-_1]A^+$ and $[\ctxm^-_2]\C^+$. By reasoning in the same way as the principal type for the application was found, we find that $\xi([\ctxm^-_1 \intersection \ctxm^-_2](\alpha \E \delta))$ with $\xi$ solving the constraint set $\{A^+_1 \le  (\C^+ \hto (\alpha \E \delta))\}$ is the typing scheme of the with.

\paragraph{Return} A return $\ret v$ is typed with the \textsc{Ret} rule from Figure~\ref{fig:core-reform-typing}. If $v$ cannot be typed, then $\ret v$ is not typeable either. Otherwise, $v$ has principal typing scheme $[\ctxm^-]A^+$. The return takes $v$ and adds an empty dirt. This means that $\ret v$ has typing scheme $[\ctxm^-](A^+ \E \emptyrow)$. Considering $\emptyrow$ is also principal in nature, $[\ctxm^-](A^+ \E \emptyrow)$ is the principal typing scheme of $\ret v$.

\paragraph{Operation} An operation $\op \, v$ is typed with the \textsc{Do} rule from Figure~\ref{fig:core-reform-typing}. If $v$ cannot be typed, or the operation $\op$ does not exist, $\op \, v$ is untypeable. If $v$ is typeable, it has the principal typing scheme $[\ctxm^-]A^+$ while an operation $\op$ has type $A^+ \to B^-$. An operation \textbf{always has a principal type}. $\op \, v$ is typed with the resulting type of the operation, with the dirt being the operation. This yields the typing scheme $[\ctxm^-](B^+ \E \op)$. All components of this typing scheme are principal.

\subsection{Principality for Booleans}
Here we discuss principality of booleans and conditionals. In Figure~\ref{fig:inference:expressions} and Figure~\ref{fig:inference:computations}, these are represented by the inference rules \textsc{True}, \textsc{False} and \textsc{Cond}.

\paragraph{Booleans} The booleans, $\tru$ and $\fls$ are typed with the \textsc{True} and \textsc{False} rules from Figure~\ref{fig:core-reform-typing}. These rules are facts and give the type $bool$ to the values. considering $bool$ is already principal, no additional work is required.

\paragraph{Conditionals} A conditional $\conditional{v}{c_1}{c_2}$ is typed with the \textsc{Cond} rule from Figure~\ref{fig:core-reform-typing}. If any of $v$, $c_1$, $c_2$ are untypeable, then so is the conditional. If they are all typeable, they respectively have principal types $[\ctxm^-_1]A^+$, $[\ctxm^-_2]\C^+_1$ and $[\ctxm^-_3]\C^+_2$.

By reasoning in the same way as the principal type for the application was found, we find that $\xi([\ctxm^-_1 \intersection \ctxm^-_2 \intersection \ctxm^-_3](\alpha \E \delta))$ with $\xi$ solving the constraint set $\{A^+ \le bool, \C^+_1 \le (\alpha \E \delta), \C^+_2 \le (\alpha \E \delta)\}$ is the typing scheme of the conditional.

\subsection{Principality of Let-Binding}
Finally, we discuss principality of let-bound variables, let-bindings and do-bindings. In Figure~\ref{fig:inference:expressions} and Figure~\ref{fig:inference:computations}, these are represented by the inference rules \textsc{Var-$\ctxp$}, \textsc{Let} and \textsc{Do}.

\paragraph{Let-bound variables} A let-bound variable $\letvar$ is typed with the \textsc{Var-$\ctxp$} rule from Figure~\ref{fig:core-reform-typing}. If $\letvar$ is not part of $\ctxp$, then $\letvar$ is not typeable. Otherwise, $\letvar$ is typeable with the principal typing scheme $\ctxp(\letvar)$

\paragraph{Let-bindings} A conditional $\letin{\letvar = v} c$ is typed with the \textsc{Let} rule from Figure~\ref{fig:core-reform-typing}. If $v$ is not typeable, then $\letin{\letvar = v} c$ is not typeable either. Otherwise, $v$ has principal type $[\ctxm^-_1]A^+$. 

The general form of a typing derivation for $\letin{\letvar = v} c$ is:

$\inferrule*[left=Let]{
    \inferrule*[left=SubVal]{
        \ctxp \entt v \T [\ctxm_1^-]A^+
      }{
        \ctxp \entt v \T [\ctxm_1]A
      }\\
      \ctxp, \letvar \T [\ctxm_1]A \entt c \T [\ctxm_2](B \E \dirt)
}{
    \ctx \entt \letin{\letvar = v} c \T [\ctxm_1 \intersection \ctxm_2](B \E \dirt)
}$

\textsc{SubVal} can only be applied if $[\ctxm_1^-]A^+ \le^\forall [\ctxm_1]A$. With the weakening (\textsc{WeakeningType} from Figure~\ref{fig:core-scheme}), we get $\ctxp, \letvar \T [\ctxm_1^-]A^+ \entt c \T [\ctxm_2](B \E \dirt)$. With this, we can apply the inductive hypothesis to this case. The principal type of $c$ is $[\ctxm^-_2](B^+ \E \dirt^+)$. Thus, the principal type of $\letin{\letvar = v} c$ is straightforward: $[\ctxm^-_1 \intersection \ctxm^-_2](B^+ \E \dirt^+)$.

\paragraph{Do-bindings} A do-binding $\doin{\letvar = c_1} c_2$ is typed with the \textsc{Do} rule from Figure~\ref{fig:core-reform-typing}. The do-binding is rather interesting. The reasoning behind finding the principal type is a combination of the conditional and the let-binding. If $c_1$ is not typeable, then $\doin{\letvar = c_1} c_2$ is not typeable either. Otherwise, $c_1$ has principal type $[\ctxm^-_1](A^+ \E \dirt^+_1)$.

The general form of a typing derivation for $\doin{\letvar = c_1} c_2$ is:

$\inferrule*[left=Let]{
    \inferrule*[left=SubComp]{
        \ctxp \entt c_1 \T [\ctxm_1^-](A^+ \E \dirt^+_1)
      }{
        \ctxp \entt c_1 \T [\ctxm_1](A \E \dirt)
      }\\
      \inferrule*[left=SubComp]{
        \ctxp, \letvar \T [\ctxm_1]A \entt c_2 \T [\ctxm_2](B \E \dirt_2)
      }{
        \ctxp, \letvar \T [\ctxm_1]A \entt c_2 \T [\ctxm_2](B \E \dirt)
      }
}{
    \ctx \entt \doin{\letvar = c_1} c_2 \T [\ctxm_1 \intersection \ctxm_2](B \E \dirt)
}$

\textsc{SubComp} can only be applied if $[\ctxm_1^-](A^+ \E \dirt^+_1) \le^\forall [\ctxm_1](A \E \dirt)$. With the weakening (\textsc{WeakeningType} from Figure~\ref{fig:core-scheme}), we get $\ctxp, \letvar \T [\ctxm_1^-](A^+ \E \dirt^+_1) \entt c \T [\ctxm_2](B \E \dirt_2)$. With this, we can apply the inductive hypothesis to this case. The principal type of $c_2$ is $[\ctxm^-_2](B^+ \E \dirt^+_2)$.

For the let-binding, we could now find the principal type. However, we still have an issue with the $\dirt$. By reasoning in the same way as the principal type for the application and conditional was found, we find that $\xi([\ctxm^-_1 \intersection \ctxm^-_2](B^+ \E \delta))$ with $\xi$ solving the constraint set $\{\dirt^+_1 \le \delta, \dirt^+_2 \le \delta\}$ is the typing scheme of the do-binding.


\begin{figure}[!htb]
\begin{center}
\begin{framed}
\begin{minipage}[t]{0.95\columnwidth}
\[\begin{array}{r@{~}c@{~}l}
    \text{monomorphic typing contexts}~\ctxm^- & \bnfis {} & \epsilon \bnfor \ctxm^-, x : A^-\\
    \text{polymorphic typing contexts}~\ctxp & \bnfis {} & \epsilon \bnfor \ctxp, \letvar : [\ctxm^-]A^+\\
    \end{array}\]
    \textbf{Expressions}
    \begin{mathpar}
    \inferrule[Var-$\ctxm$]{
    }{
        \ctxp \prinent x \T [x : \alpha]\alpha
    }
    
    \inferrule[Var-$\ctxp$]{
        (\letvar \T [\ctxm^-]A^+) \in \ctxp
    }{
        \ctxp \prinent \letvar \T [\ctxm^-]A^+
    }    
    
    \inferrule[True]{
    }{
        \ctxp \prinent \tru \T []bool
    }
    
    \inferrule[False]{
    }{
        \ctxp \prinent \fls \T []bool
    }
    
    \inferrule[Fun]{
        \ctxp \prinent c \T [\ctxm^-]\C^+
    }{
        \ctxp \prinent \fun{x}c \T [\ctxm^- \setminus x](\ctxm^-(x) \to \C^+)
    }
    
    \inferrule[Hand]{
        \ctxp \prinent c_r \T [\ctxm^-_r](B^+ \E \dirt^+) \\
        \Big[
        (\op \T A^+_\op \to B^-_\op) \in \sig \qquad
        \ctxp \prinent c_\op \T [\ctxm^-_\op](\C^+_\op)
        \Big]_{\op \in \ops}
    }{
        \ctxp \prinent \shorthand \T \\
        [\ctxm^-_r \intersection (\bigsqcap \ctxm^-_\op | \op \in \ops)](\ctxm^-_r(x) \E \delta \intersection \ops \hto \alpha \E \delta)
    }\\ \xi = biunify\left(
        \begin{array}{ll}
            B^+ \E \dirt^+ \le \alpha \E \delta \\
            % \alpha_1 \le \ctxm^-_r(x) \\
            % \delta_1 \le \delta_2 \\
            \left[
            \begin{array}{ll}
                A^+_\op \le \ctxm^-_\op(y) \\
                B^-_\op \to (\alpha \E \delta) \le \ctxm^-_\op(k) \\
                \C^+_\op \le \alpha \E \delta
            \end{array}
            \right]_{\op \in \ops}
        \end{array}
        \right)
\end{mathpar}
\end{minipage}
\end{framed}
\end{center}
\caption{Type inference algorithm for expressions}\label{fig:inference:expressions}
\end{figure}

\begin{figure}[!htb]
\begin{center}
\begin{framed}
\begin{minipage}[t]{0.95\columnwidth}
\[\begin{array}{r@{~}c@{~}l}
    \text{monomorphic typing contexts}~\ctxm^- & \bnfis {} & \epsilon \bnfor \ctxm^-, x : A^-\\
    \text{polymorphic typing contexts}~\ctxp & \bnfis {} & \epsilon \bnfor \ctxp, \letvar : [\ctxm^-]A^+\\
    \end{array}\]
    \textbf{Computations}
    \begin{mathpar}
    \inferrule[App]{
        \ctxp \prinent v_1 \T [\ctxm^-_1]A^+_1 \\
        \ctxp \prinent v_2 \T [\ctxm^-_2]A^+_2
    }{
        \ctxp \prinent v_1 \, v_2 \T \xi([\ctxm^-_1 \intersection \ctxm^-_2](\alpha \E \delta))
    } \xi = biunify(A^+_1 \le A^+_2 \to (\alpha \E \delta))
    
    \inferrule[Cond]{
        \ctxp \prinent v \T [\ctxm^-_1]A^+ \\
        \ctxp \prinent c_1 \T [\ctxm^-_2]\C^+_1 \\
        \ctxp \prinent c_2 \T [\ctxm^-_3]\C^+_2 \\
    }{
        \ctxp \prinent \conditional{v}{c_1}{c_2} \T \xi([\ctxm^-_1 \intersection \ctxm^-_2 \intersection \ctxm^-_3](\alpha \E \delta))
    } \xi = biunify\left(
        \begin{array}{ll}
            A^+ \le bool\\
            \C^+_1 \le (\alpha \E \delta)\\
            \C^+_2 \le (\alpha \E \delta)
        \end{array}
      \right)
    
    \inferrule[Ret]{
        \ctxp \prinent v \T [\ctxm^-]A^+
    }{
        \ctxp \prinent \ret v \T [\ctxm^-](A^+ \E \emptyrow)
    }
    
    \inferrule[Op]{
        (\op \T A^+ \to B^-) \in \sig \\
        \ctxp \prinent v \T [\ctxm^-]A^+
    }{
        \ctxp \prinent \op \, v \T [\ctxm^-](B^+ \E \op)
    }
    
    \inferrule[Let]{
        \ctxp \prinent v \T [\ctxm^-_1]A^+ \\
        \ctxp, \letvar \T [\ctxm^-_1]A^+ \prinent c \T [\ctxm^-_2](B^+ \E \dirt^+)
    }{
        \ctx \prinent \letin{\letvar = v} c \T [\ctxm^-_1 \intersection \ctxm^-_2](B^+ \E \dirt^+)
    }

    \inferrule[Do]{
        \ctxp \prinent c_1 \T [\ctxm^-_1](A^+ \E \dirt^+_1) \\
        \ctxp, \letvar \T [\ctxm^-_1]A^+ \prinent c_2 \T [\ctxm^-_2](B^+ \E \dirt^+_2)
    }{
        \ctx \prinent \doin{\letvar = c_1} c_2 \T \xi([\ctxm^-_1 \intersection \ctxm^-_2](B^+ \E \delta))
    } \xi = biunify\left(
        \begin{array}{ll}
            \dirt^+_1 \le \delta\\
            \dirt^+_2 \le \delta
        \end{array}
    \right)
    
    \inferrule[With]{
        \ctxp \prinent v \T [\ctxm^-_1]A^+ \\
        \ctxp \prinent c \T [\ctxm^-_2]\C^+
    }{
        \ctxp \prinent \withhandle{v}{c} \T \xi([\ctxm^-_1 \intersection \ctxm^-_2](\alpha \E \delta))
    } \xi = biunify(A^+ \le \C^+ \hto (\alpha \E \delta))
    
\end{mathpar}
\end{minipage}
\end{framed}
\end{center}
\caption{Type inference algorithm for computations}\label{fig:inference:computations}
\end{figure}