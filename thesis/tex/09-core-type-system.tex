\section{Equivalence to Subtyping}\label{typesystem}

Algebraic subtyping is equivalent to standard subtyping constraints. \cite{dolan2017algebraic} Figure~\ref{fig:core-relation} shows the equivalence rules between the different subtyping constraints and algebraic subtyping. $A_1 \le A_2 \leftrightarrow A_1 \union A_2 \equiv A_2$ shows that if a type $A_1$ is a subtype of $A_2$, then $A_1 \union A_2$ is equivalent to $A_2$. This rule and the analogue of intersection are part of algebraic subtyping. The novel additions are $\dirt_1 \le \dirt_2 \leftrightarrow \dirt_1 \union \dirt_2 \equiv \dirt_2$ and $\dirt_1 \le \dirt_2 \leftrightarrow \dirt_1 \equiv \dirt_1 \intersection \dirt_2$. These rules show equivalence for dirts.

Figure~\ref{fig:core-equations-types} are equations of distributive lattices for types. These are entirely standard for algebraic subtyping. Figure~\ref{fig:core-equations-dirts} show the distributive lattices for dirts, which follow the format that algebraic subtyping uses for types. It is an important detail that the equations for the dirts mirror those of types, as we want the effect system to mirror the algebraic subtyping for types as much as possible.

Figure~\ref{fig:core-equations-other-types} shows equivalence rules for functions, handlers and dirty types. Equivalence rules regarding functions are exactly like they are in standard algebraic subtyping. Functions are contravariant in its argument types and covariant in its return types. This can also be seen in these equivalence equations. When we have a union of two function types, the argument types are intersected. When we have an intersection of two function types,the argument types are unioned. Handlers are also contravariant in its argument types and covariant in its return types, and thus show the same behaviour. 

The final two rules for dirty types $(\C_1 \union \C_2) \equiv (A_1 \E \dirt_1 \union A_2 \E \dirt_2) \equiv (A_1 \union A_2) \E (\dirt_1 \union \dirt_2)$ and $(\C_1 \intersection \C_2) \equiv (A_1 \E \dirt_1 \intersection A_2 \E \dirt_2) \equiv (A_1 \intersection A_2) \E (\dirt_1 \intersection \dirt_2)$ show the decomposition of dirty types into its two components, a type and a dirt. The union of two dirty types is equivalent to the union of its types and its dirts, and analogue for the intersection. 

\begin{figure}[H]
\begin{center}
\begin{framed}
\begin{minipage}[t]{0.95\columnwidth}
\begin{mathpar}    
    \inferrule[]{}{
        A_1 \le A_2 \leftrightarrow A_1 \union A_2 \equiv A_2
    }\\

    \inferrule[]{}{
        A_1 \le A_2 \leftrightarrow A_1 \equiv A_1 \intersection A_2
    }\\

    \inferrule[]{}{
        \dirt_1 \le \dirt_2 \leftrightarrow \dirt_1 \union \dirt_2 \equiv \dirt_2
    }\\

    \inferrule[]{}{
        \dirt_1 \le \dirt_2 \leftrightarrow \dirt_1 \equiv \dirt_1 \intersection \dirt_2
    }\\

    \inferrule[]{}{
        \C_1 \le \C_2 \leftrightarrow \C_1 \union \C_2 \equiv \C_2
    }\\

    \inferrule[]{}{
        \C_1 \le \C_2 \leftrightarrow \C_1 \equiv \C_1 \intersection \C_2
    }
\end{mathpar}
\end{minipage}
\end{framed}
\end{center}
\caption{Relationship between Equivalence and Subtyping}\label{fig:core-relation}
\end{figure}


\begin{figure}[H]
\begin{center}
\begin{framed}
% \framebox{
\begin{minipage}[t]{0.475\columnwidth}
% \textbf{Equations of distributive lattices for types}
\begin{mathpar}
    \inferrule[]{}{
      A \union A \equiv A
    }\\
    
    \inferrule[]{}{
      A_1 \union A_2 \equiv A_2 \union A_1
    }\\

    \inferrule[]{}{
      A_1 \union (A_2 \union A_3) \equiv (A_1 \union A_2) \union A_3
    }\\

    \inferrule[]{}{
      A_1 \union (A_1 \intersection A_2) \equiv A_1
    }\\

    \inferrule[]{}{
      \bot \union A \equiv A
    }\\

    \inferrule[]{}{
      \top \union A \equiv \top
    }
\end{mathpar}
\end{minipage}
\begin{minipage}[t]{0.475\columnwidth}
\begin{mathpar}    
    \inferrule[]{}{
        A \intersection A \equiv A
    }\\

    \inferrule[]{}{
        A_1 \intersection A_2 \equiv A_2 \intersection A_1
    }\\

    \inferrule[]{}{
        A_1 \intersection (A_2 \intersection A_3) \equiv (A_1 \intersection A_2) \intersection A_3
    }\\

    \inferrule[]{}{
        A_1 \intersection (A_1 \union A_2) \equiv A_1
    }\\

    \inferrule[]{}{
        \bot \intersection A \equiv \bot
    }\\

    \inferrule[]{}{
        \top \intersection A \equiv A
    }
\end{mathpar}
\end{minipage}
\begin{minipage}[t]{0.95\columnwidth}
\begin{mathpar}    
    \inferrule[]{}{
        A_1 \union (A_2 \intersection A_3) \equiv (A_1 \union A_2) \intersection (A_1 \union A_3)
    }\\

    \inferrule[]{}{
        A_1 \intersection (A_2 \union A_3) \equiv (A_1 \intersection A_2) \union (A_1 \intersection A_3)
    }
\end{mathpar}
\end{minipage}
% }
\end{framed}
\end{center}
\caption{Equations of distributive lattices for types}\label{fig:core-equations-types}
\end{figure}

\begin{figure}[H]
\begin{center}
\begin{framed}
% \framebox{
\begin{minipage}[t]{0.475\columnwidth}
\begin{mathpar}
    \inferrule[]{}{
        \dirt \union \dirt \equiv \dirt
    }\\
    
    \inferrule[]{}{
        \dirt_1 \union \dirt_2 \equiv \dirt_2 \union \dirt_1
    }\\

    \inferrule[]{}{
        \dirt_1 \union (\dirt_2 \union \dirt_3) \equiv (\dirt_1 \union \dirt_2) \union \dirt_3
    }\\

    \inferrule[]{}{
        \dirt_1 \union (\dirt_1 \intersection \dirt_2) \equiv \dirt_1
    }\\

    \inferrule[]{}{
        \emptyrow \union \dirt \equiv \dirt
    }\\

    \inferrule[]{}{
        \allops \union \dirt \equiv \allops
    }
\end{mathpar}
\end{minipage}
\begin{minipage}[t]{0.475\columnwidth}
\begin{mathpar}    
    \inferrule[]{}{
        \dirt \intersection \dirt \equiv \dirt
    }\\

    \inferrule[]{}{
        \dirt_1 \intersection \dirt_2 \equiv \dirt_2 \intersection \dirt_1
    }\\

    \inferrule[]{}{
        \dirt_1 \intersection (\dirt_2 \intersection \dirt_3) \equiv (\dirt_1 \intersection \dirt_2) \intersection \dirt_3
    }\\

    \inferrule[]{}{
        \dirt_1 \intersection (\dirt_1 \union \dirt_2) \equiv \dirt_1
    }\\

    \inferrule[]{}{
        \emptyrow \intersection \dirt \equiv \emptyrow
    }\\

    \inferrule[]{}{
        \allops \intersection \dirt \equiv \dirt
    }
\end{mathpar}
\end{minipage}
\begin{minipage}[t]{0.95\columnwidth}
\begin{mathpar}    
    \inferrule[]{}{
        \dirt_1 \union (\dirt_2 \intersection \dirt_3) \equiv (\dirt_1 \union \dirt_2) \intersection (\dirt_1 \union \dirt_3)
    }\\

    \inferrule[]{}{
        \dirt_1 \intersection (\dirt_2 \union \dirt_3) \equiv (\dirt_1 \intersection \dirt_2) \union (\dirt_1 \intersection \dirt_3)
    }
\end{mathpar}
\end{minipage}
% }
\end{framed}
\end{center}
\caption{Equations of distributive lattices for dirts}\label{fig:core-equations-dirts}
\end{figure}

\begin{figure}[H]
\begin{center}
\begin{framed}
\begin{minipage}[t]{0.95\columnwidth}
\begin{mathpar}    
    \inferrule[]{}{
        (A_1 \to \C_1) \union (A_2 \to \C_2) \equiv (A_1 \intersection A_2) \to (\C_1 \union \C_2)
    }\\

    \inferrule[]{}{
        (A_1 \to \C_1) \intersection (A_2 \to \C_2) \equiv (A_1 \union A_2) \to (\C_1 \intersection \C_2)
    }\\

    \inferrule[]{}{
        (A_1 \hto \C_1) \union (A_2 \hto \C_2) \equiv (A_1 \intersection A_2) \hto (\C_1 \union \C_2)
    }\\

    \inferrule[]{}{
        (A_1 \hto \C_1) \intersection (A_2 \hto \C_2) \equiv (A_1 \union A_2) \hto (\C_1 \intersection \C_2)
    }\\
    
    \inferrule[]{}{
        (\C_1 \union \C_2) \equiv (A_1 \E \dirt_1 \union A_2 \E \dirt_2) \equiv (A_1 \union A_2) \E (\dirt_1 \union \dirt_2)
    }\\

    \inferrule[]{}{
        (\C_1 \intersection \C_2) \equiv (A_1 \E \dirt_1 \intersection A_2 \E \dirt_2) \equiv (A_1 \intersection A_2) \E (\dirt_1 \intersection \dirt_2)
    }
\end{mathpar}
\end{minipage}
\end{framed}
\end{center}
\caption{Equations for function, handler and dirty types}\label{fig:core-equations-other-types}
\end{figure}