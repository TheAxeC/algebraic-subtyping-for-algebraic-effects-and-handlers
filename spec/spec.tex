\documentclass[sigplan,10pt]{acmart}\settopmatter{printfolios=true}
% \documentclass[acmsmall,10pt]{acmart}\settopmatter{printfolios=true}
%% For final camera-ready submission
%\documentclass[sigplan,10pt]{acmart}\settopmatter{}

\settopmatter{printacmref=false} % Removes citation information below abstract
\renewcommand\footnotetextcopyrightpermission[1]{} % removes footnote with conference information in first column
\pagestyle{plain} % removes running headers

%% Some recommended packages.
\usepackage{booktabs}   %% For formal tables:
                        %% http://ctan.org/pkg/booktabs
\usepackage{subcaption} %% For complex figures with subfigures/subcaptions
                        %% http://ctan.org/pkg/subcaption

\usepackage{array}
\usepackage{mathpartir}
\usepackage{xspace}
\usepackage{stmaryrd}
\usepackage{listings}
\usepackage{newtxmath}

%% Tikz Needed packages
\usepackage{pgfplots}
\pgfplotsset{width=7cm,compat=1.8}
\usepackage{pgfplotstable}
\renewcommand*{\familydefault}{\sfdefault}

%\makeatletter\if@ACM@journal\makeatother
%%% Journal information (used by PACMPL format)
%%% Supplied to authors by publisher for camera-ready submission
%\acmJournal{ICFP 17 student research competition}
%\acmVolume{1}
%\acmNumber{1}
%\acmArticle{1}
%\acmYear{2017}
%\acmMonth{1}
%\acmDOI{10.1145/nnnnnnn.nnnnnnn}
%\startPage{1}
%\else\makeatother
%%% Conference information (used by SIGPLAN proceedings format)
%%% Supplied to authors by publisher for camera-ready submission
%\acmConference[ICFP 2017 Student Research Competition]{ICFP 2017 Student Research Competition}{September, 2017}{Oxford, UK}
%\acmYear{2017}
%\acmISBN{978-x-xxxx-xxxx-x/YY/MM}
%\acmDOI{10.1145/nnnnnnn.nnnnnnn}
%\startPage{1}
%\fi

%% Copyright information
%% Supplied to authors (based on authors' rights management selection;
%% see authors.acm.org) by publisher for camera-ready submission
\setcopyright{none}             %% For review submission
%\setcopyright{acmcopyright}
%\setcopyright{acmlicensed}
%\setcopyright{rightsretained}
%\copyrightyear{2017}           %% If different from \acmYear

%% Bibliography style
\bibliographystyle{ACM-Reference-Format}
%% Citation style
%% Note: author/year citations are required for papers published as an
%% issue of PACMPL.
%\citestyle{acmauthoryear}  %% For author/year citations
%\citestyle{acmnumeric}     %% For numeric citations
%\setcitestyle{nosort}      %% With 'acmnumeric', to disable automatic
                            %% sorting of references within a single citation;
                            %% e.g., \cite{Smith99,Carpenter05,Baker12}
                            %% rendered as [14,5,2] rather than [2,5,14].
%\setcitesyle{nocompress}   %% With 'acmnumeric', to disable automatic
                            %% compression of sequential references within a
                            %% single citation;
                            %% e.g., \cite{Baker12,Baker14,Baker16}
                            %% rendered as [2,3,4] rather than [2-4].

\newcommand{\lang}{\textsc{Effy}\xspace}
\newcommand{\eff}{\textsc{Eff}\xspace}
\newcommand{\core}{\textsc{EffCore}\xspace}
\newcommand{\ocaml}{\textsc{OCaml}\xspace}

% Meta-syntax
\newcommand{\bnfis}{\mathrel{\;{:}{:}\!=}\;}
\newcommand{\bnfor}{\mathrel{\;|\;}}
\newcommand{\defeq}{\mathrel{\;\stackrel{\text{def}}{=}\;}}
\newcommand{\set}[1]{\{ #1 \}}

% General syntactic constructs
\newcommand{\kord}[1]{\mathtt{#1}}
\newcommand{\kop}[1]{\;\mathtt{#1}\;}
\newcommand{\kpre}[1]{\mathtt{#1}\;}
\newcommand{\kpost}[1]{\;\mathtt{#1}}

% Types
\newcommand{\type}[1]{\mathtt{#1}}
\newcommand{\boolty}{\type{bool}}
\newcommand{\intty}{\type{int}}
\newcommand{\hto}{\Rightarrow}
\newcommand{\C}{\underline{C}}
\newcommand{\D}{\underline{D}}
\newcommand{\dirt}{\Delta}
\newcommand{\dirtend}{. (\textsc{dot})}
\newcommand{\sig}{\Sigma}

% Expressions and computations
\newcommand{\funtyped}[3]{\kpre{fun} #1 \T #2 \mapsto #3}

\newcommand{\union}{\sqcup}
\newcommand{\intersection}{\sqcap}
\newcommand{\emptyrow}{\{.\}}
\newcommand{\polytype}[1]{\forall \alpha . #1}

\newcommand{\call}[3]{{{#1}\,{#2}\,{#3}}}
\newcommand{\case}{\mathop{\text{\texttt{|}}}}
\newcommand{\cont}[2]{(#1.\,#2)}
\newcommand{\const}{\kord{k}}
\newcommand{\fls}{\kord{false}}
\newcommand{\fun}[1]{\kpre{fun} #1 \mapsto}
\newcommand{\handler}[1]{\{ #1 \}}
\newcommand{\conditional}[3]{\kpre{if} #1 \kop{then} #2 \kop{else} #3}
\newcommand{\letin}[1]{\kpre{let} #1 \kop{in}}
\newcommand{\doin}[1]{\kpre{do} #1 \kop{ ; }}
\newcommand{\letrecin}[1]{\kpre{let} \kpre{rec} #1 \kop{in}}
\newcommand{\op}{\kord{Op}}
\newcommand{\ops}{\mathcal{O}}
\newcommand{\ocs}{\mathit{ocs}}
\newcommand{\ocsnil}{\kord{nil}}
\newcommand{\tru}{\kord{true}}
\newcommand{\ret}{\kpre{return}}
\newcommand{\withhandle}[2]{\kpre{handle} #2 \kop{with} #1}
\newcommand{\pure}[1]{\kord{pure } #1  }
\newcommand{\longcases}{\call{\op_1}{y}{k} \mapsto c_{\op_1}, \ldots, \call{\op_n}{x}{k} \mapsto c_{\op_n}}
\newcommand{\shortcases}{[\call{\op}{y}{k} \mapsto c_\op]_{\op \in \ops}}
\newcommand{\longhand}[1][\ret x \mapsto c_r]{\handler{#1, \longcases}}
\newcommand{\shorthand}[1][\ret x \mapsto c_r]{\handler{#1, \shortcases}}
\newcommand{\shorthandelab}[1][\ret x \mapsto c'_r]{\handler{#1, \shortcaseselab}}
\newcommand{\shortcaseselab}{[\call{\op}{y}{k} \mapsto c'_\op]_{\op \in \ops}}

% Type-checking
\newcommand{\row}{\mathrel{;} R}
\newcommand{\ctx}{\Gamma}
\newcommand{\ent}{\vdash}
\newcommand{\T}{\mathrel{:}}
\newcommand{\E}{\mathrel{!}}
\newcommand{\covers}{\mathrel{/}}
\renewcommand{\le}{\leqslant}

% Operational semantics
\newcommand{\eval}{\Downarrow}
\newcommand{\hs}{\mathcal{H}}
\newcommand{\nil}{\emptyset}
\newcommand{\cons}{\mathbin{::}}
\newcommand{\hseval}[1][\hs]{\Downarrow_{#1}}
\newcommand{\getval}[1]{{#1}_{\kord{val}}}
\newcommand{\getop}[1]{{#1}_{\kord{op}}}

\newcommand{\todo}[1]{\textcolor{red}{\textsc{Todo:} #1}}

\begin{document}

\title{Algebraic subtyping for algebraic effects and handlers}

\author{Axel Faes}
\affiliation{
  \position{student : undergraduate\\ACM Student Member: 2461936}
  \department{Department of Computer Science}
  \institution{KU Leuven}
}
\email{axel.faes@student.kuleuven.be}

\author{Amr Hany Saleh}
\affiliation{
  \position{daily advisor}
  \department{Department of Computer Science}
  \institution{KU Leuven}
}
\email{amrhanyshehata.saleh@kuleuven.be }

\author{Tom Schrijvers}
\affiliation{
  \position{promotor}
  \department{Department of Computer Science}
  \institution{KU Leuven}
}
\email{tom.schrijvers@kuleuven.be}

%% Keywords
%% comma separated list
\keywords{algebraic effect handler, algebraic subtyping, effects, optimised compilation}  %% \keywords is optional

%% Abstract
%% Note: \begin{abstract}...\end{abstract} environment must come
%% before \maketitle command
\begin{abstract}
Algebraic effects and handlers are a very active area of research. An important aspect is the development of an optimising compiler. \eff is an ML-style language with support for effects and forms the testbed for the optimising compiler. However, the type-\&-effect system of \eff is unsatisfactory. This is due to the lack of some elegant properties. It is also awkward to implement and use in practice.
\end{abstract}

\maketitle

\tableofcontents

\listoffigures
\listoftables

\section{Introduction}
The specification for a type-\&-effect system with algebraic subtyping for algebraic effects and handlers is given in this document. The formal properties of this system are studied in order to find which properties are satisfied compared to other type-\&-effect systems. The proposed type-\&-effect system builds on two very recent developments in the area of programming language theory.

\paragraph{Algebraic subtyping}
In his December 2016 PhD thesis, Stephen Dolan (University of Cambridge, UK), has presented a novel type system that combines subtyping and parametric polymorphism in a particulary attractive and elegant fashion. A cornerstone of his design are the algebraic properties that the subtyping relation should respect.

\paragraph{Algebraic effects and handlers}
These are a new formalism for formally modelling side-effects (e.g. mutable state or non-determinism) in programming languages, developed by Matija Pretnar (University of Ljubjana) and Gordon Plotkin (University of Edinburgh). This approach is gaining a lot of traction, not only as a formalism but also as a practical feature in actual programming languages (e.g. the Koka language developed by Microsoft Research). We are collaborating with Matija Pretnar on the efficient implementation of one such language, called Eff. Axel Faes has contributed to this collaboration during a project he did for the Honoursprogramme of the Faculty of Engineering Science.

\subsection{Motivation}
Algebraic effects and handlers benefit from a custom type-\&-effect system, a type system that also tracks which effects can happen in a program. Several such type-\&-effect systems have been proposed in the literature, but all are unsatisfactory. We attribute this to the lack of the elegant properties of Dolan's type system. Indeed the existing type-\&-effect systems are not only theoretically unsatisfactory, but they are also awkward to implement and use in practice.

\paragraph{Research questions}
\begin{itemize}
\item How can Dolan's elegant type system be extended with effect information?
\item Which properties are preserved and which aren't preserved?
\item What advantages are there to an type-\&-effect system based on Dolan's elegant type system?
\end{itemize}

\subsection{Goals}
The goal of this thesis is to derive a type-\&-effect system that extends Dolan's elegant type system with effect information. This type-\&-effect system should inherit Dolan's harmonious combination of subtyping (in our case induced by a lattice structure on the effect information) with parametric polymorphism and preserve all of its desirable properties (both low-level algebraic properties and high-level meta-theoretical properties like type soundness and the existence of principal types). Afterwards this type-\&-effect system The following approach is taken:
\begin{enumerate}
\item Study of the relevant literature and theoretical background.
\item Design of a type-\&-effect system derived from Dolan's, that integrates effects.
\item Proving the desirable properties of the proposed type-\&-effect system: type soundness, principal typing, ...
\item Time permitting: Design of a type inference algorithm that derives the principal types of programs without type annotations and proving its correctness.
\item Time permitting: Implementation of the algorithm and comparing it to other algorithms (such as row polymorphism based type-\&-effect systems).
\end{enumerate}


\section{Background}
\input{tex/background}

\section{Related Work (\eff)}
The type-\&-effect system that is used in \eff is based on subtyping and dirty types \cite{effectsystem}.

\subsection{Types and terms}

\paragraph{Terms}
Figure~\ref{fig:terms:eff} shows the two types of terms in \eff. There are values $v$ and computations $c$. Computations are terms that can contain effects. Effects are denoted as operations $Op$ which can be called.

\begin{figure}[!htb]
\begin{center}
\framebox{
\begin{minipage}{0.98\columnwidth}
\[\begin{array}{r@{~}c@{~}l@{\quad}l}
  \text{value}~v & \bnfis {} & x & \text{variable} \\
    & \bnfor & \const & \text{constant} \\
    & \bnfor & \fun{x} c & \text{function} \\
    & \bnfor & \{ & \text{handler} \\
    & & \quad \ret x \mapsto c_r, & \quad\text{return case} \\
    & & \quad \shortcases & \quad\text{operation cases} \\
    & & \} & \\
  \text{comp}~c & \bnfis & v_1 \, v_2 & \text{application} \\
    & \bnfor & \letrecin{f \, x = c_1} c_2 & \text{rec definition} \\
    & \bnfor & \ret v  & \text{returned val} \\
    & \bnfor & \op \, v & \text{operation call} \\
    & \bnfor & \doin{x \leftarrow c_1} c_2 & \text{sequencing} \\
    & \bnfor & \withhandle{v}{c} & \text{handling}
\end{array}\]
\end{minipage}
}
\end{center}
\caption{Terms of \eff}\label{fig:terms:eff}
\end{figure}

\paragraph{Types}
Figure~\ref{fig:types:eff} shows the types of \eff. There are two main sorts of types. There are (pure) types $A, B$ and dirty types $\C, \D$. A dirty type is a pure type $A$ tagged with a finite set of operations $\dirt$, which we call dirt, that can be called. This finite set $\dirt$ is an over-approximation of the operations that are actually called. The type $\C \hto \D$ is used for handlers because a handler takes an input computation $\C$, handles the effects in this computation and outputs computation $\D$ as the result.

\begin{figure}[!htb]
\begin{center}
\framebox{
\begin{minipage}{0.98\columnwidth}
\[\begin{array}{r@{~}c@{~}l@{\quad}l}
  \text{(pure) type}~A, B & \bnfis {}
    & \boolty \bnfor \intty & \text{basic types} \\
    & \bnfor & A \to \C & \text{function type} \\
    & \bnfor & \C \hto \D & \text{handler type} \\
  \text{dirty type}~\C, \D & \bnfis {} & A \E \dirt \\
  \text{dirt}~\dirt & \bnfis {} &\set{\op_1, \dots, \op_n}
\end{array}\]
\end{minipage}
}
\end{center}
\caption{Types of \eff}\label{fig:types:eff}
\end{figure}

\subsection{Type System}
\subsubsection{Subtyping}
The dirty type $A \E \dirt$ is assigned to a computation returning values of type $A$ and potentially calling operations from the set $\dirt$. This set $\dirt$ is always an over-approximation of the actually called operations, and may safely be increased, inducing a natural subtyping judgement $A \E \dirt \leq A \E \dirt'$ on dirty types. As dirty types can occur inside pure types, we also get a derived subtyping judgement on pure types. Both judgements are defined in Figure~\ref{fig:subtyping}. Observe that, as usual, subtyping is contravariant in the argument types of functions and handlers, and covariant in their return types.

\begin{figure}[!htb]
\begin{center}
\framebox{
\begin{minipage}{0.95\columnwidth}
\textbf{Subtyping}
\begin{mathpar}
  \inferrule[Sub-$\boolty$]{
  }{
    \boolty \le \boolty
  }

  \inferrule[Sub-$\intty$]{
  }{
    \intty \le \intty
  }

  \inferrule[Sub-$\to$]{
    A' \le A \\
    \C \le \C'
  }{
    A \to \C \le A' \to \C'
  }

  \inferrule[Sub-$\hto$]{
    \C' \le \C \\
    \D \le \D'
  }{
    \C \hto \D \le \C' \hto \D'
  }

  \inferrule[Sub-$\E$]{
    A \le A' \\
    \dirt \subseteq \dirt'
  }{
    A \E \dirt \le A' \E \dirt'
  }
\end{mathpar}
\end{minipage}
}
\end{center}
\caption{Subtyping for pure and dirty types of \eff}\label{fig:subtyping}
\end{figure}

\subsubsection{Typing rules}
Figure~\ref{fig:eff-typing} defines the typing judgements for values and computations with respect to a standard typing context $\ctx$.

\paragraph{Values}
The rules for subtyping, variables, and functions are entirely standard. For constants we assume a signature $\sig$ that assigns a type~$A$ to each constant~$\const$, which we write as $(\const \T A) \in \sig$.

A handler expression has type $A \E \dirt \cup \ops \hto B \E \dirt$ iff all branches (both the operation cases and the return case) have dirty type $B \E \dirt$ and the operation cases cover the set of operations $\ops$. Note that the intersection $\dirt \cap \ops$ is not necessarily empty. The handler deals with the operations $\ops$, but in the process may re-issue some of them (i.e., $\dirt \cap \ops$).

When typing operation cases, the given signature for the operation $(\op \T A_\op \to B_\op) \in \sig$ determines the type $A_\op$ of the parameter $x$ and the domain $B_\op$ of the continuation $k$. As our handlers are deep, the codomain of $k$ should be the same as the type $B \E \dirt$ of the cases.

\paragraph{Computations}
With the following exceptions, the typing judgement $\ctx \ent c : \C$ has a straightforward definition. The $\ret$ construct renders a value $v$ as a pure computation, i.e., with empty dirt. An operation invocation $\op\,v$ is typed according to the operation's signature, with the operation itself as its only operation. Finally, rule \textsc{With} shows that a handler with type $\C \hto \D$ transforms a computation with type $\C$ into a computation with type $\D$.

\begin{figure}[!htb]
\begin{center}
\framebox{
\begin{minipage}{0.95\columnwidth}
\[\begin{array}{r@{~}c@{~}l}
  \text{typing contexts}~\ctx & \bnfis {} & \epsilon \bnfor \ctx, x : A\\
\end{array}\]
\textbf{Expressions}
\begin{mathpar}
  \inferrule[SubVal]{
    \ctx \ent v \T A \\
    A \le A'
  }{
    \ctx \ent v \T A'
  }

  \inferrule[Var]{
    (x \T A) \in \ctx
  }{
    \ctx \ent x \T A
  }

  \inferrule[Const]{
    (\const \T A) \in \sig
  }{
    \ctx \ent \const \T A
  }

  \inferrule[Fun]{
    \ctx, x \T A \ent c \T \C
  }{
    \ctx \ent \fun{x} c \T A \to \C
  }

  \inferrule[Hand]{
    \ctx, x \T A \ent c_r \T B \E \dirt \\
    \Big[
      (\op \T A_\op \to B_\op) \in \sig \qquad \\
      \ctx, x \T A_\op, k \T B_\op \to B \E \dirt \ent c_\op \T B \E \dirt
    \Big]_{\op \in \ops}
  }{
    \ctx \ent \shorthand \T \\ A \E \dirt \cup \ops \hto B \E \dirt
  }
\end{mathpar}
\textbf{Computations}
\begin{mathpar}
  \inferrule[SubComp]{
    \ctx \ent c \T \C \\
    \C \le \C'
  }{
    \ctx \ent c \T \C'
  }

  \inferrule[App]{
    \ctx \ent v_1 \T A \to \C \\
    \ctx \ent v_2 \T A
  }{
    \ctx \ent v_1 \, v_2 \T \C
  }

 \inferrule[LetRec]{
    \ctx, f \T A \to \C, x \T A \ent c_1 \T \C \\
    \ctx, f \T A \to \C \ent c_2 \T \D
  }{
    \ctx \ent \letrecin{f \, x = c_1} c_2 \T \D
  }

  \inferrule[Ret]{
    \ctx \ent v \T A
  }{
    \ctx \ent \ret v \T A \E \emptyset
  }

  \inferrule[Op]{
    (\op \T A \to B) \in \sig \\
    \ctx \ent v \T A
  }{
    \ctx \ent \op \, v \T B \E \{\op\}
  }

  \inferrule[Do]{
    \ctx \ent c_1 \T A \E \dirt \\
    \ctx, x \T A \ent c_2 \T B \E \dirt
  }{
    \ctx \ent \doin{x \leftarrow c_1} c_2 \T B \E \dirt
  }

  \inferrule[With]{
    \ctx \ent v \T \C \hto \D \\
    \ctx \ent c \T \C
  }{
    \ctx \ent \withhandle{v}{c} \T \D
  }
\end{mathpar}
\end{minipage}
}
\end{center}
\caption{Typing of \eff}\label{fig:eff-typing}
\end{figure}


\section{Core language (\core)}
% The core language with row-based effects is based on the explicitly typed language used in Links \cite{row}. Links uses a row polymorphic type-\&-effect system . The design of their calculus is partially based on the type system used by Pretnar which makes it a suitable candidate for our core language \cite{pretnar2015introduction}. The terms of the core language are seen in Figure~\ref{fig:terms:explicit}, the types are seen in the Figure~\ref{fig:types:explicit}.
$\sqcup$ is for outputs, $\sqcap$ is for inputs.

\subsection{Types and terms}
The proposed type system uses type application and type abstractions in order to attain explicit typing information. The subtyping approach is replaced by polymorphism. Aside from this, the type system remains as close as possible to the source language of Eff in order to maximise compatability.

\begin{figure}[h]
\begin{center}
\framebox{
\begin{minipage}{0.98\columnwidth}
\[\begin{array}{r@{~}c@{~}l@{\quad}l}
  \text{value}~v & \bnfis {} & x & \text{variable} \\
    & \bnfor & \const & \text{constant} \\
    & \bnfor & \lambda (x : A). c & \text{\textbf{function}} \\
    & \bnfor & \Lambda \alpha . c & \text{\textbf{type abstraction}} \\
    & \bnfor & \{ & \text{handler} \\
    & & \quad \ret x \mapsto c_r, & \quad\text{return case} \\
    & & \quad \shortcases & \quad\text{operation cases} \\
    & & \} & \\
  \text{comp}~c & \bnfis & v_1 \, v_2 & \text{application} \\
    & \bnfor & v \, A & \text{\textbf{type application}} \\
    & \bnfor & \letrecin{f \, x = c_1} c_2 & \text{rec definition} \\
    & \bnfor & \ret v  & \text{returned val} \\
    & \bnfor & \op \, v & \text{operation call} \\
    & \bnfor & \doin{x \leftarrow c_1} c_2 & \text{sequencing} \\
    & \bnfor & \withhandle{v}{c} & \text{handling}
\end{array}\]
\end{minipage}
}
\end{center}
\caption{Terms of the explicitly typed core language}\label{fig:terms:explicit}
\end{figure}

\begin{figure}
\begin{center}
\framebox{
\begin{minipage}{0.98\columnwidth}
\[\begin{array}{r@{~}c@{~}l@{\quad}l}
  \text{(pure) type}~A, B & \bnfis {}
    & A \to \C & \text{function type} \\
    & \bnfor & \C \hto \D & \text{handler type} \\
    & \bnfor & \alpha & \text{\textbf{type variable}} \\
    & \bnfor & \forall \alpha . \C & \text{\textbf{polytype}} \\
  \text{dirty type}~\C, \D & \bnfis {} & A \E \dirt \\
  \text{dirt}~\dirt & \bnfis {} & \{\text{R}\} \\
  \text{R} & \bnfis {}
    & \op \row & \text{row} \\
    & \bnfor & \delta & \text{row variable} \\
    & \bnfor & . & \text{end of row}
\end{array}\]
\end{minipage}
}
\end{center}
\caption{Types of the explicitly type core language}\label{fig:types:explicit}
\end{figure}

\subsection{Typing rules}
There are no surprises in the typing rules either. The typing rules are standard rules. It is important to note the row polymorphism in the \textit{Hand} rule. 

\begin{figure}
\begin{center}
\framebox{
\begin{minipage}{0.95\columnwidth}
\[\begin{array}{r@{~}c@{~}l}
  \text{typing contexts}~\ctx & \bnfis {} & \epsilon \bnfor \ctx, x : A\\
\end{array}\]
\textbf{Expressions}
\begin{mathpar}
  \inferrule[Val]{
  }{
    \ctx \ent v \T A
  }

  \inferrule[Var]{
    (x \T A) \in \ctx
  }{
    \ctx \ent x \T A
  }

  \inferrule[Const]{
    (\const \T A) \in \sig
  }{
    \ctx \ent \const \T A
  }

  \inferrule[Fun]{
    \ctx, x \T A \ent c \T \C
  }{
    \ctx \ent \lambda(x \T A). c \T A \to \C
  }

  \inferrule[Type Abstraction]{
    \ctx, \alpha \ent c \T \C
  }{
    \ctx \ent \Lambda \alpha . c \T \forall \alpha . \C
  }

  \inferrule[Hand]{
  	\C = A \E \{Op_i \row\} \\
  	\D = B \E \{Op_i \row\} \\
  	(\op_i \T A_\op \to B_\op) \in \sig \qquad \\
  	h = \shorthand \\
  	\ctx, x \T A_\op \ent c_r \T \D \\
  	\ctx, y \T A_\op, k \T B_\op \to \D \ent c_{op} \T \D \\
  }{
  	\ctx \ent h \T \C \hto \D
  }
\end{mathpar}
\textbf{Computations}
\begin{mathpar}
  \inferrule[Comp]{
  }{
    \ctx \ent c \T \C
  }

  \inferrule[App]{
    \ctx \ent v_1 \T A \to \C \\
    \ctx \ent v_2 \T A
  }{
    \ctx \ent v_1 \, v_2 \T \C
  }

  \inferrule[Type App]{
    \ctx \ent v \T \forall \alpha . \C
  }{
    \ctx \ent v \, A \T \C[A/\alpha]
  }

 \inferrule[LetRec]{
    \ctx, f \T A \to \C, x \T A \ent c_1 \T \C \\
    \ctx, f \T A \to \C \ent c_2 \T \D
  }{
    \ctx \ent \letrecin{f \, x = c_1} c_2 \T \D
  }

  \inferrule[Ret]{
    \ctx \ent v \T A
  }{
    \ctx \ent \ret v \T A \E \emptyset
  }

  \inferrule[Op]{
    (\op \T A \to B) \in \sig \\
    \ctx \ent v \T A \\
    \C \T B \E \{\op \row\}
  }{
    \ctx \ent \op \, v \T \C
  }

  \inferrule[Do]{
    \ctx \ent c_1 \T A \E \dirt \\
    \ctx, x \T A \ent c_2 \T B \E \dirt
  }{
    \ctx \ent \doin{x \leftarrow c_1} c_2 \T B \E \dirt
  }

  \inferrule[With]{
    \ctx \ent v \T \C \hto \D \\
    \ctx \ent c \T \C
  }{
    \ctx \ent \withhandle{v}{c} \T \D
  }
\end{mathpar}
\end{minipage}
}
\end{center}
\caption{Typing of the explicitly typed language}\label{fig:core-typing}
\end{figure}


\section{Semantics}
\input{tex/semantics}

\section{Elaboration}
The elaboration show how the source language can be transformed into \core.

\begin{figure}[!htb]
\begin{center}
\framebox{
\begin{minipage}{0.95\columnwidth}
\[\begin{array}{r@{~}c@{~}l}
  \text{typing contexts}~\ctx & \bnfis {} & \epsilon \bnfor \ctx, x : A \bnfor \ctx,  x : \polytype{B}\\
\end{array}\]
\textbf{Expressions}
\begin{mathpar}
  \inferrule[Val]{
    \ctx \ent v \T A \leadsto v'\\
    \ent_c A \le B
  }{
    \ctx \ent v \T B \leadsto v'
  }

  \inferrule[Var]{
    (x \T S) \in \ctx \\
    S = \forall \bar{\alpha} . A
  }{
    \ctx \ent x \T A[\bar{S}/\bar{\alpha}] \leadsto x \, \bar{S}
  }

  \inferrule[Const]{
    (\const \T A) \in \sig
  }{
    \ctx \ent \const \T A \leadsto \const'
  }

  \inferrule[Fun]{
    \ctx, x \T A \ent c \T \C \leadsto c'
  }{
    \ctx \ent \fun{x} c \T A \to \C \leadsto \funtyped{x}{A}{c'} \T A \to \C
  }

  \inferrule[Hand]{
    \ctx, x \T A \ent c_r \T B \E \dirt \\
    \Big[
      (\op \T A_\op \to B_\op) \in \sig \qquad \\
      \ctx, x \T A_\op, k \T B_\op \to B \E \dirt \ent c_\op \T B \E \dirt
    \Big]_{\op \in \ops}
  }{
    \ctx \ent \shorthand \T \\ A \E \dirt \cup \ops \hto B \E \dirt \\ \leadsto \shorthand \T \\ A \E \dirt \cup \ops \hto B \E \dirt
  }

\end{mathpar}
\end{minipage}
}
\end{center}
\caption{Elaboration of source to core language: expressions}\label{fig:elaboration:exp}
\end{figure}


\begin{figure}[!htb]
\begin{center}
\framebox{
\begin{minipage}{0.95\columnwidth}
\[\begin{array}{r@{~}c@{~}l}
  \text{typing contexts}~\ctx & \bnfis {} & \epsilon \bnfor \ctx, x : A, x : \polytype{B}\\
\end{array}\]
\textbf{Computations}
\begin{mathpar}
  \inferrule[Comp]{
    \ctx \ent c \T \C \leadsto c'\\
    \C \le \D
  }{
    \ctx \ent c \T \D \leadsto c'
  }

  \inferrule[App]{
    \ctx \ent v_1 \T A \to \C \leadsto v'_1 \\
    \ctx \ent v_2 \T A \leadsto v'_2
  }{
    \ctx \ent v_1 \, v_2 \T \C \leadsto v'_1 \, v'_2 \T \C
  }

 \inferrule[LetRec]{
    \ctx, f \T A \to \C, x \T A \ent c_1 \T \C \leadsto c'_1 s\\
    \ctx, f \T A \to \C \ent c_2 \T \D \leadsto c'_2
  }{
    \ctx \ent \letrecin{f \, x = c_1} c_2 \T \D \\ \leadsto \letrecin{f \, x = c'_1} c'_2 \T \D
  }

  \inferrule[Ret]{
    \ctx \ent v \T A \leadsto v'
  }{
    \ctx \ent \ret v \T A \E \emptyset \leadsto \ret v' \T A \E \emptyrow
  }

  \inferrule[Op]{
    (\op \T A \to B) \in \sig \\
    \ctx \ent v \T A \leadsto v'
  }{
    \ctx \ent \op \, v \T B \E \{\op\} \leadsto \op \, v' \T B \E \{\op, .\}
  }

  \inferrule[Do]{
    \ctx \ent c_1 \T \C \leadsto c'_1 \\
    S = \forall \bar{\alpha} . A \\
    \bar{\alpha} = FTV(A) - TV(\ctx) \\
    \ctx, x \T S \ent c_2 \T \D \leadsto c'_2
  }{
    \begingroup\color{red}
    \ctx \ent \doin{x \leftarrow c_1} c_2 \T \D \leadsto (\funtyped{x}{A}{c'_2})(\Lambda \bar{\alpha} . c'_1)
    \endgroup
  }

  \inferrule[With]{
    \ctx \ent v \T \C \hto \D \leadsto v' \\
    \ctx \ent c \T \C \leadsto c'
  }{
    \ctx \ent \withhandle{v}{c} \T \D \leadsto \withhandle{v'}{c'} \T \D
  }
\end{mathpar}
\end{minipage}
}
\end{center}
\caption{Elaboration of source to core language: computations}\label{fig:elaboration:comp}
\end{figure}


\section{Proofs}
\input{tex/proofs}

\section{Implementation}
\input{tex/background}

\section{Evaluation}
\input{tex/background}

\section{Conclusion}
Algebraic effects and handlers are a very active area of research. An important aspect is the development of an optimising compiler. Without a type-\&-effect system with explicit typing, it is easy for type checking bugs to be introduced during the construction of optimised compilation. A core language with row-based effects was introduced. The core language is explicitly typed in order to reduce bugs in the optimised compilation.


\appendix
\section{Appendix A}
\input{tex/appendix}

%% Acknowledgments
\begin{acks}
  I would like to thank Amr Hany Saleh for his continuous guidance and help.
\end{acks}

\bibliography{bib/main}

\end{document}
