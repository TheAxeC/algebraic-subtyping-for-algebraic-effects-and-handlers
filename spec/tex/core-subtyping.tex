\subsection{Subtyping rules}
The subtyping rules are given in Figure~\ref{fig:core-subtyping}, Figure~\ref{fig:core-subtyping-dist} and Figure~\ref{fig:core-subtyping-dirt}. Figure~\ref{fig:core-subtyping} contains all subtyping rules related to types. Figure~\ref{fig:core-subtyping-dist} contains the distributative subtyping rules. Finally Figure~\ref{fig:core-subtyping-dirt} contains the subtyping rules that govern the dirts.

The dirty type $A \E \dirt$ is assigned to a computation returning values of type $A$ and potentially calling operations from the set $\dirt$. This set $\dirt$ is always an over-approximation of the actually called operations, and may safely be increased, inducing a natural subtyping judgement $A \E \dirt \leq A \E \dirt'$ on dirty types. As dirty types can occur inside pure types, we also get a derived subtyping judgement on pure types. Observe that, as usual, subtyping is contravariant in the argument types of functions and handlers, and covariant in their return types.

\paragraph{Dirt intersection and union}
There are several possible methods to compute the dirt intersection and union. If row variables were to be disregarded, dirt union and intersection could be defined as set union and intersection. This methods allows unions and intersections to be eliminated. This has an advantage, eliminating unions and intersections simplifies the effect system. However, we cannot disregard row variables.

Thus, set union and intersection cannot simply be used. It would be possible to define $\delta_1 \union \delta_2$ and $\delta_1 \intersection \delta_2$. Using these, it is possible to use a form of set union and intersection. The following union $\{Op_1, ..., Op_n, \delta_1\} \union \{Op_{n+1}, ..., Op_{n+m}, \delta_2\}$ could be defined as $\{Op_1, ..., Op_n, Op_{n+1}, ..., Op_{n+m}, (\delta_1 \union \delta_2)\}$. A similar construction can be used for intersection. This simplifies the subtyping rules since the more complicated aspects are enclosed within the row variables. The equivalence rules are defined in Figure~\ref{fig:core-equivalence}.

\begin{figure}[!htb]
\begin{center}
\framebox{
\begin{minipage}{0.95\columnwidth}
\textbf{Subtyping of pure and dirty types}
\begin{mathpar}
  \inferrule[Sub-Top]{
  }{
    A \le \top
  }

  \inferrule[Sub-Bottom]{
  }{
    \bot \le A
  }
  \\

  \inferrule[Sub-$\boolty$]{
  }{
    \boolty \le \boolty
  }

  % \inferrule[Sub-$\intty$]{
  % }{
  %   \intty \le \intty
  % }
  \\

  \inferrule[Sub-Refl]{
  }{
    A \le A
  }

  \inferrule[Sub-Trans]{
    A_1 \le A_2 \\
    A_2 \le A_3
  }{
    A_1 \le A_3
  }
  \\

  \inferrule[Sub-$\E$]{
    A_1 \le A'_2 \\
    \dirt_1 \le \dirt'_2
  }{
    A_1 \E \dirt_1 \le A_2 \E \dirt_2
  }

  % \inferrule[Sub-All]{
  %   A \le B
  % }{
  %  \polytype{A} \le \polytype{B}
  % }
  \\

  \inferrule[Sub-$\to$]{
    A_2 \le A_1 \\
    \C_1 \le \C_2
  }{
    A_1 \to \C_1 \le A_2 \to \C_2
  }

  \inferrule[Sub-$\hto$]{
    \C_2 \le \C_1 \\
    \D_1 \le \D_2
  }{
    \C_1 \hto \D_1 \le \C_2 \hto \D_2
  }
  \\
  % \inferrule[Sub-Inter]{
  %   A_1 \le A_2 \\
  %   B_1 \le B_2 \\
  % }{
  %   A_1 \intersection B_1 \le A_2 \intersection B_2
  % }

  % \inferrule[Sub-Union]{
  %   A_1 \le A_2 \\
  %   B_1 \le B_2 \\
  % }{
  %   A_1 \union B_1 \le A_2 \union B_2
  % }

  % \inferrule[Sub-Inter-Greatest-LB]{
  %   A \le B_1 \\
  %   A \le B_2 \\
  % }{
  %   A \le B_1 \intersection B_2
  % }

  % \inferrule[Sub-Inter-LB]{
  %   i \in \{1; 2\}
  % }{
  %   A_1 \intersection A_2 \le A_i
  % }

  % \inferrule[Sub-Union-Least-UB]{
  %   B_1 \le A \\
  %   B_2 \le A \\
  % }{
  %   B_1 \union B_2 \le A
  % }

  % \inferrule[Sub-Inter-UB]{
  %   i \in \{1; 2\}
  % }{
  %   A_i \le A_1 \union A_2
  % }
\end{mathpar}
\end{minipage}
}
\end{center}
\caption{Subtyping for pure and dirty types of \core}\label{fig:core-subtyping}
\end{figure}

\begin{figure}[!htb]
\begin{center}
\framebox{
\begin{minipage}{0.95\columnwidth}
\textbf{Subtyping of dirts}
\begin{mathpar}
  \inferrule[Sub-$\E$-Empty]{
  }{
    \emptyrow \le \dirt
  }

  \inferrule[Sub-$\E$-Top]{
  }{
     \dirt \le \allops
  }

  \inferrule[Sub-$\E$-Trans]{
    \dirt_1 \le \dirt_2 \\
    \dirt_2 \le \dirt_3
  }{
     \dirt_1 \le \dirt_3
  }

  \inferrule[Sub-$\E$-Row]{
  }{
     \{Op_1, ..., Op_n, .\} \le \{Op_1, ..., Op_n, \delta\}
  }

  \inferrule[Sub-$\E$-Refl]{
  }{
     \dirt \le \dirt
  }



  \inferrule[Sub-$\E$-Row-Row]{
    n \ge 0 \\
    m \ge 0 \\
    p \ge 0 \\
    \{Op_1, ..., Op_{n}, Op_{n+m+1}, ..., Op_{n+m+p}, \delta_1\} \le \\ \{Op_1, ..., Op_n, Op_{n+1}, ..., Op_{n+m}, \delta_2\}
  }{
    \{\delta_1\} \le \{Op_{n+1}, ..., Op_{n+m}, \delta_2\} \\
    \{\delta_2\} = \{Op_{n+m}, ..., Op_{n+m+p}, \delta_3\}
  }

  \inferrule[Sub-$\E$-Dot-Row]{
    n \ge 0 \\
    m \ge 0 \\
    p \ge 0 \\
    \{Op_1, ..., Op_{n}, Op_{n+m+1}, ..., Op_{n+m+p}, .\} \le \\ \{Op_1, ..., Op_n, Op_{n+1}, ..., Op_{n+m}, \delta_2\}
  }{
    \emptyrow \le \{Op_{n+1}, ..., Op_{n+m}, \delta_2\} \\
    \{\delta_2\} = \{Op_{n+m}, ..., Op_{n+m+p}, \delta_3\}
  }

  \inferrule[Sub-$\E$-Row-Dot]{
    n \ge 0 \\
    m \ge 0 \\
    \{Op_1, ..., Op_n, \delta_1\} \le \{Op_1, ..., Op_n, Op_{n+1}, Op_{n+m}, .\}
  }{
    \{\delta_1\} \le \{Op_{n+1}, Op_{n+m}, .\}
  }

  \inferrule[Sub-$\E$-Dot-Dot]{
    n \ge 0 \\
    m \ge 0 \\
    \{Op_1, ..., Op_n, .\} \le \{Op_1, ..., Op_n, Op_{n+1}, ..., Op_{n+m}, .\}
  }{
    \emptyrow \le \{Op_{n+1}, Op_{n+m}, .\}
  }

  \inferrule[Sub-Inter-$\E$]{
    \dirt_1 \le \dirt_2 \\
    \dirt_3 \le \dirt_4 \\
    \dirt_1 \ne \dirt_3 \\
  }{
    \dirt_1 \intersection \dirt_3 \le \dirt_2 \intersection \dirt_4
  }

  \inferrule[Sub-Union-$\E$]{
    \dirt_1 \le \dirt_2 \\
    \dirt_3 \le \dirt_4 \\
    \dirt_1 \ne \dirt_3 \\
  }{
    \dirt_1 \union \dirt_3 \le \dirt_2 \union \dirt_4
  }

  \inferrule[Sub-Inter-Greatest-LB-$\E$]{
    \dirt_1 \le \dirt_2 \\
    \dirt_1 \le \dirt_3
  }{
    \dirt_1 \le (\dirt_2 \intersection \dirt_3)
  }

  \inferrule[Sub-Inter-LB-$\E$]{
    i \in \{1; 2\}
  }{
    (\dirt_1 \intersection \dirt_2) \le \dirt_i
  }

  \inferrule[Sub-Union-Least-UB-$\E$]{
    \dirt_2 \le \dirt_1 \\
    \dirt_3 \le \dirt_1
  }{
    (\dirt_2 \union \dirt_3) \le \dirt_1
  }

  \inferrule[Sub-Union-UB-$\E$]{
    i \in \{1; 2\}
  }{
    \dirt_i \le (\dirt_1 \union \dirt_2)
  }

  % \inferrule[Sub-Inter-Union-$\E$]{
  % }{
  %   (\dirt_1 \intersection \dirt_3) \union (\dirt_2 \intersection \dirt_4) \le (\dirt_1 \union \dirt_2) \intersection (\dirt_3 \union \dirt_4)
  % }
  %
  % \inferrule[Sub-Union-Inter-$\E$]{
  % }{
  %   (\dirt_1 \union \dirt_3) \intersection (\dirt_2 \union \dirt_4) \le (\dirt_1 \intersection \dirt_2) \union (\dirt_3 \intersection \dirt_4)
  % }
\end{mathpar}
\end{minipage}
}
\end{center}
\caption{Subtyping for dirts of \core}\label{fig:core-subtyping-dirt}
\end{figure}
