Algebraic effect handling is a very active area of research. Implementations of algebraic effect handlers are becoming available and the theory is actively being developed. The type-\&-effect system that is used in \eff is based on subtyping and algebraic effect handlers \cite{effectsystem}. The \textit{simply typed lambda calculus} is used as a basis for \eff.  Let us start with a simple example in order to show what algebraic effects and handlers are. With this example, the differences with the \textit{simply typed lambda calculus} can also be shown.

In the example below, a new effect is defined \textit{DivisionByZero}. In essence, this effect can be thought of as an exception. From the type that is written, it can also be seen that an exception has some relation with functions. In this case, the effect describes a function type from \textit{unit} to \textit{empty}. This type describes what kind of argument the effect requires in order to be called and what kind of type it leaves behind after being handled. 

\begin{lstlisting}[language=Caml]
effect DivisionByZero : unit -> empty;;

let divide a b = 
  if (b == 0) then 
    #DivisionByZero () 
  else 
    a / b;;

let safeDivide a b = 
  handle (divide a b) with 
    | #DivisionByZero () k -> 0;;
\end{lstlisting}

The effect can be called just like any function can be called, by applying an argument to it. Here, an important distinction can be made. Any term that can contain effects are called computations and are dirty, while terms that cannot contain effects are called expressions and are pure. Finally, computations can be handled. This can be thought of as an exception handler with the big difference being that within an effect handler, there is access to a continuation to the place where the effect was called. 

To reiterate, in order to extend \textit{simply typed lambda calculus} to \eff, several terms need to be added. A term is required in order to call effects and handle effects. Of course, we need to be able to have handlers as well. \cite{pretnar2015introduction}



