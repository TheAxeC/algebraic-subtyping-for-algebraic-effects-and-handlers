% The elaboration show how the source language can be transformed into the core language. Most rules are straightforward except for the \textit{Do} rule.
%
% \begin{figure}
% \begin{center}
% \framebox{
% \begin{minipage}{0.95\columnwidth}
% \[\begin{array}{r@{~}c@{~}l}
%   \text{typing contexts}~\ctx & \bnfis {} & \epsilon \bnfor \ctx, x : A\\
% \end{array}\]
% \textbf{Expressions}
% \begin{mathpar}
%   \inferrule[Val]{
%   }{
%     \ctx \ent v \T A \leadsto v'
%   }
%
%   \inferrule[Var]{
%     (x \T S) \in \ctx \\
%     S = \forall \bar{\alpha} . A
%   }{
%     \ctx \ent x \T A[\bar{S}/\bar{\alpha}] \leadsto x \, \bar{S'}
%   }
%
%   \inferrule[Const]{
%     (\const \T A) \in \sig
%   }{
%     \ctx \ent \const \T A \leadsto \const'
%   }
%
%   \inferrule[Fun]{
%     \ctx, x \T A \ent c \T \C \leadsto c'
%   }{
%     \ctx \ent \fun{x} c \T A \to \C \leadsto \lambda (x \T A) . c' \T A \to \C
%   }
%
%   % \inferrule[Hand]{
%   %   \ctx, x \T A \ent c_r \T B \E \dirt \\
%   %   \Big[
%   %     (\op \T A_\op \to B_\op) \in \sig \qquad \\
%   %     \ctx, x \T A_\op, k \T B_\op \to B \E \dirt \ent c_\op \T B \E \dirt
%   %   \Big]_{\op \in \ops}
%   % }{
%   %   \ctx \ent \shorthand \T \\ A \E \dirt \cup \ops \hto B \E \dirt
%   % }
%
%   \inferrule[Hand]{
%   	\C = A \E \{Op_i \row\} \\
%   	\D = B \E \{Op_i \row\} \\
%   	(\op_i \T A_\op \to B_\op) \in \sig \qquad \\
%   	h = \shorthand \\ \leadsto h' = \shorthandelab \\
%   	\ctx, x \T A_\op \ent c_r \T \D \leadsto c'_r \T \D \\
%   	\ctx, y \T A_\op, k \T B_\op \to \D \ent c_{op} \T \D \leadsto c'_{op} \T \D\\
%   }{
%   	\ctx \ent h \T \C \hto \D \leadsto h' \T \C \hto \D
%   }
% \end{mathpar}
% \end{minipage}
% }
% \end{center}
% \caption{Elaboration of source to core language: expressions}\label{fig:elaboration:exp}
% \end{figure}
%
%
% \begin{figure}
% \begin{center}
% \framebox{
% \begin{minipage}{0.95\columnwidth}
% \[\begin{array}{r@{~}c@{~}l}
%   \text{typing contexts}~\ctx & \bnfis {} & \epsilon \bnfor \ctx, x : A\\
% \end{array}\]
% \textbf{Computations}
% \begin{mathpar}
%   \inferrule[Comp]{
%   }{
%     \ctx \ent c \T \C' \leadsto c'
%   }
%
%   \inferrule[App]{
%     \ctx \ent v_1 \T A \to \C \leadsto v'_1 \\
%     \ctx \ent v_2 \T A \leadsto v'_2
%   }{
%     \ctx \ent v_1 \, v_2 \T \C \leadsto v'_1 \, v'_2 \T \C
%   }
%
%  \inferrule[LetRec]{
%     \ctx, f \T A \to \C, x \T A \ent c_1 \T \C \leadsto c'_1 s\\
%     \ctx, f \T A \to \C \ent c_2 \T \D \leadsto c'_2
%   }{
%     \ctx \ent \letrecin{f \, x = c_1} c_2 \T \D \\ \leadsto \letrecin{f \, x = c'_1} c'_2 \T \D
%   }
%
%   \inferrule[Ret]{
%     \ctx \ent v \T A \leadsto v'
%   }{
%     \ctx \ent \ret v \T A \E \emptyset \leadsto \ret v' \T A \E \emptyset
%   }
%
%   \inferrule[Op]{
%     (\op \T A \to B) \in \sig \\
%     \C = B \E \{\op \row \} \\
%     \ctx \ent v \T A \leadsto v'
%   }{
%     \ctx \ent \op \, v \T \C \leadsto \op \, v' \T \C
%   }
%
%   \inferrule[Do]{
%     \ctx \ent c_1 \T \C \leadsto c'_1 \\
%     S = \forall \bar{\alpha} . A \\
%     \bar{\alpha} = FTV(A) - TV(\ctx) \\
%     \ctx, x \T S \ent \D \leadsto c'_2
%   }{
%     \ctx \ent \doin{x \leftarrow c_1} c_2 \T \D \leadsto (\lambda (x \T A) . c'_2) (\Lambda \bar{\alpha} . c'_1)
%   }
%
%   \inferrule[With]{
%     \ctx \ent v \T \C \hto \D \leadsto v' \\
%     \ctx \ent c \T \C \leadsto c'
%   }{
%     \ctx \ent \withhandle{v}{c} \T \D \leadsto \withhandle{v'}{c'} \T \D
%   }
% \end{mathpar}
% \end{minipage}
% }
% \end{center}
% \caption{Elaboration of source to core language: computations}\label{fig:elaboration:comp}
% \end{figure}
