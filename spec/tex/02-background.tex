\subsection{Programming language theory}
The field of programming language theory is a branch of computer science that describes how to formaly define complete programming languages and programming language features, such as algebraic effect handlers.

The work described in this thesis uses several aspects from programming language theory. An important subdiscipline that is extensively used is type theory. Type theory is used to formaly describe type systems. A type system is a set of rules that are used to define the shape of meaningful programs. The \textit{simply typed lambda calculus} will be used to show and explain the necessary background that is required for further chapters. The \textit{simply typed lambda calculus} is the simplest and most elementary form of all typed languages. \cite{pierce2002types}

\subsection{Types and terms}

\paragraph{Terms}
Figure~\ref{fig:terms:lambda} shows the three sorts of terms of the \textit{simply typed lambda calculus}. A variable by itself is already a term. The abstraction of some variable $x$ from a certain term $t$ is called a function. Finally, an application is a term. The terms define the syntax of a programming language, but it does not place any constraints on how these terms can be composed. A wanted constraint could for example be that an application $t_1 \, t_2$ should only be valid if $t_1$ is a function. This shows that only having terms is not enough to describe a programming language. \cite{hindley1986introduction}

\begin{figure}[!htb]
\begin{center}
\framebox{
\begin{minipage}{0.98\columnwidth}
\[\begin{array}{r@{~}c@{~}l@{\quad}l}
  \text{terms}~t & \bnfis {} & x & \text{variable} \\
    & \bnfor & \tru & \text{true} \\
    & \bnfor & \fls & \text{false} \\
    & \bnfor & \fun{x \T T} t & \text{function} \\
    & \bnfor & t_1 \, t_2 & \text{application}
\end{array}\]
\end{minipage}
}
\end{center}
\caption{Terms of simply typed lambda calculus}\label{fig:terms:lambda}
\end{figure}

\paragraph{Types}
Since the \textit{simply typed lambda calculus} is being used, types are needed. As seen in figure~\ref{fig:types:lambda}, there are two types, the base type and the function type. In a valid and meaningful program, every term has a type. A term is called well typed or typable if there is a type for that term. 

\begin{figure}[!htb]
\begin{center}
\framebox{
\begin{minipage}{0.98\columnwidth}
\[\begin{array}{r@{~}c@{~}l@{\quad}l}
  \text{type}~T & \bnfis {}
    & bool & \text{base type} \\
    & \bnfor & T \to T & \text{function type}
\end{array}\]
\end{minipage}
}
\end{center}
\caption{Types of simply typed lambda calculus}\label{fig:types:lambda}
\end{figure}

\subsection{Typing rules}

As stated above, a method is needed to place constraints on the programming language. This is done with typing rules or types judgements. The typing rules for the \textit{simply typed lambda calculus} are given in figure~\ref{fig:lambda-typing}. 

\begin{figure}[!htb]
\begin{center}
\framebox{
\begin{minipage}{0.95\columnwidth}
\[\begin{array}{r@{~}c@{~}l}
  \text{typing contexts}~\ctx & \bnfis {} & \epsilon \bnfor \ctx, x : T \\
\end{array}\]
\begin{mathpar}
  \inferrule[True]{
  }{
    \ctx \ent \tru \T bool
  }

  \inferrule[False]{
  }{
    \ctx \ent \fls \T bool
  }

  \inferrule[Var]{
    (x \T T) \in \ctx
  }{
    \ctx \ent x \T T
  }

  \inferrule[App]{
    \ctx \ent t_1 \T T_1 \to T_2 \\
    \ctx \ent t_2 \T T_1
  }{
    \ctx \ent t_1 \, t_2 \T T_2
  }
  
  \inferrule[Fun]{
    \ctx, x \T T_1 \ent t \T T_2
  }{
    \ctx \ent \fun{x \T T} t \T T_1 \to T_2
  }
\end{mathpar}
\end{minipage}
}
\end{center}
\caption{Typing of simply typed lambda calculus}\label{fig:lambda-typing}
\end{figure}

The first rules to take note of are the \textsc{True} and \textsc{False} rules. These are facts and state that the terms true and false have type \textit{bool}. A \textit{Fact} states that, under the assumption of $\ctx$, $t$ has type $T$. The context, $\ctx$, is a mapping of the free variables of $t$ to their types. It is called a fact since the rule always holds.

The context, $\ctx$, is a (possibly empty) collection of variables mapped to their types. The \textsc{Var} rule states that, if the context contains a mapping for a variable, that variable is also a valid term with that type. The \textsc{App} rule defines the usage of a function. When there are two terms $t_1$ and $t_2$ with types $T_1 \to T_2$ and $T_1$, then the application $t_1 \, t_2$ will have the type $T_2$.  

An inference rule can be read in multiple ways. It can be read top-down or bottom-up. Reading it top-down gives the above described reasoning. Given some expressions and some constraints, another expression can be constructed with a specific type. The bottom-up approach states that, given an expression such as the function application, there is a specific way the different parts of the expression can be typed. In the \textsc{App} rule, a function expression has type $T_2$. Therefor, both $t_1$ and $t_2$ must follow a specific set of constraints. It is known that a function needs to exist of type $T_1 \to T_2$ and an expression that matches the argument of the function, $T_1$ needs to exist. \cite{pierce2002types}

Finally, there is the \textsc{Fun} rule. This rule is also called a function abstraction or simply an abstraction. It shows how a function can be constructed. The interesting part of this rule is $\ctx, x \T T_1 \ent t \T T_2$. This states that $t$ is only entailed by some context and a variable of type $T_1$.

\subsection{Other extensions}

Now, a full specification of the \textit{simply typed lambda calculus} is given. However, there are many extensions that can be added onto this language. In the next chapter, \eff will be discussed. \eff is a language which can be described as a modification of the \textit{simply typed lambda calculus} with algebraic effects and handlers. \eff is also uses subtyping rules, this concept will also be further explained in the next chapter. After this, algebraic subtyping will be added to the language.

Of course, just a specification does not have much meaning. Certain aspects or properties could be proved in order to show that they do (or do not) hold in the given language. Type inference is another aspect which is not talked about in this chapter. Type inference revolves around the automatic detection (or inference) of the types of terms. Both proofs and a type inference algorithm are given in later chapters. 