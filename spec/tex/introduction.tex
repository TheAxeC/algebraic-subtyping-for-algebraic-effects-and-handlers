
\label{intro}
Algebraic effect handling is a very active area of research. Implementations of algebraic effect handlers are becoming available. Because of this, improving performance is becoming the focus of research. A lot of research focusses on speeding up the runtime performance. However, a runtime penalty still occurs. This happens since handlers or continuations need to be repeatedly copied on the heap. Due to this, we are looking towards type-directed optimised compilation of algebraic effect handlers. We want to remove the handlers such that no copying is required and thus no runtime penalty occurs. \\
\\
In our ongoing research towards type-directed optimised compilation, term rewrite rules and purity aware compilation optimise away most handlers. Term rewrite rules use information of the type-\&-effect system. Term rewrite rules perform two types of actions. They remove handlers and apply effects such that eventually the program does not contain any more handlers. Term rewrite rules can also change the syntactic structure in order to expose more possibilities for optimisations. Purity aware compilation identifies computations that are effectively pure and purifies them.  \\
\\
\eff, an ML-style language, is being used to develop an optimised compiler for algebraic effect handlers. \eff uses a type system based on subtyping \cite{effectsystem}. As explained by Bauer and Pretnar in \cite{programming}, terms in \eff do not contain any information about computational effects. This information has to be inferred using type inference algorithms. The lack of explicit type information makes source-to-source transformations much more error-prone. Additionally, ensuring that a transformation does not break typeability becomes a time-consuming task, since we need to reconstruct types after each optimisation pass. \\
\\
The current type system with subtyping becomes impractical since the typing information is not explicitly contained in each term. There are several solutions to make the type system more practical. It is possible to keep subtyping, but use a unification based algorithm \cite{mlsub}. Implicit effect polymorphism can also be used \cite{impliciteff}. The option that is explored in this work, is to use a simple type-\&-effect system based on row-polymorphism \cite{type-directed, leijen2014koka, row}. \\
\\
In this work, we present a simple explicitly-typed language that can serve as an intermediate language during compilation of \eff, and allows for the development of type-preserving core-to-core transformations. Optimisation and term rewriting is done using this core language. This approach will ease the development of an optimised compiler since typechecking becomes linear due to the explicit typing.\\
\\
Below is an example program:
\begin{lstlisting}[language=Caml]
effect Op : unit -> int;;
let rec x () = #Op ();;
let result =
  handle (x ()) with
    | #Op () k -> k 1
\end{lstlisting}
This program will be optimised into the code below after applying a function specialization optimization. The function 'x' is specialized and the handler is brought inside the function. Implementing this optimization requires indepth knowledge of the typing system that is used. The subtyping-based approach for the type system is difficult to use and does not contain explicit typing information. This makes source-to-source transformations error prone and
ensuring transformations do not break typability is time consuming.

\begin{lstlisting}[language=Caml]
effect Op : unit -> int;;
let rec x () = #Op ();;
let result =
  let rec x_spec () =
    handle (#Op ()) with
    | #Op () k -> k 1
  in
  x_spec ()
\end{lstlisting}
