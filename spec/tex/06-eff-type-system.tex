\subsection{Type System}
\subsubsection{Subtyping}
The dirty type $A \E \dirt$ is assigned to a computation returning values of type $A$ and potentially calling operations from the set $\dirt$. This set $\dirt$ is always an over-approximation of the actually called operations, and may safely be increased, inducing a natural subtyping judgement $A \E \dirt \leq A \E \dirt'$ on dirty types. As dirty types can occur inside pure types, we also get a derived subtyping judgement on pure types. Both judgements are defined in Figure~\ref{fig:subtyping}. Observe that, as usual, subtyping is contravariant in the argument types of functions and handlers, and covariant in their return types.

\begin{figure}[!htb]
\begin{center}
\framebox{
\begin{minipage}{0.95\columnwidth}
\textbf{Subtyping}
\begin{mathpar}
  \inferrule[Sub-$\boolty$]{
  }{
    \boolty \le \boolty
  }

  % \inferrule[Sub-$\intty$]{
  % }{
  %   \intty \le \intty
  % }

  \inferrule[Sub-$\to$]{
    A' \le A \\
    \C \le \C'
  }{
    A \to \C \le A' \to \C'
  }

  \inferrule[Sub-$\hto$]{
    \C' \le \C \\
    \D \le \D'
  }{
    \C \hto \D \le \C' \hto \D'
  }

  \inferrule[Sub-$\E$]{
    A \le A' \\
    \dirt \subseteq \dirt'
  }{
    A \E \dirt \le A' \E \dirt'
  }
\end{mathpar}
\end{minipage}
}
\end{center}
\caption{Subtyping for pure and dirty types of \eff}\label{fig:subtyping}
\end{figure}

\subsubsection{Typing rules}
Figure~\ref{fig:eff-typing} defines the typing judgements for values and computations with respect to a standard typing context $\ctx$.

\paragraph{Values}
The rules for subtyping, variables, and functions are entirely standard. For constants we assume a signature $\sig$ that assigns a type~$A$ to each constant~$\const$, which we write as $(\const \T A) \in \sig$.

A handler expression has type $A \E \dirt \cup \ops \hto B \E \dirt$ iff all branches (both the operation cases and the return case) have dirty type $B \E \dirt$ and the operation cases cover the set of operations $\ops$. Note that the intersection $\dirt \cap \ops$ is not necessarily empty. The handler deals with the operations $\ops$, but in the process may re-issue some of them (i.e., $\dirt \cap \ops$).

When typing operation cases, the given signature for the operation $(\op \T A_\op \to B_\op) \in \sig$ determines the type $A_\op$ of the parameter $x$ and the domain $B_\op$ of the continuation $k$. As our handlers are deep, the codomain of $k$ should be the same as the type $B \E \dirt$ of the cases.

\paragraph{Computations}
With the following exceptions, the typing judgement $\ctx \ent c : \C$ has a straightforward definition. The $\ret$ construct renders a value $v$ as a pure computation, i.e., with empty dirt. An operation invocation $\op\,v$ is typed according to the operation's signature, with the operation itself as its only operation. Finally, rule \textsc{With} shows that a handler with type $\C \hto \D$ transforms a computation with type $\C$ into a computation with type $\D$.

\begin{figure}[!htb]
\begin{center}
\framebox{
\begin{minipage}{0.95\columnwidth}
\[\begin{array}{r@{~}c@{~}l}
  \text{typing contexts}~\ctx & \bnfis {} & \epsilon \bnfor \ctx, x : A\\
\end{array}\]
\textbf{Expressions}
\begin{mathpar}
  \inferrule[SubVal]{
    \ctx \ent v \T A \\
    A \le A'
  }{
    \ctx \ent v \T A'
  }

  \inferrule[Var]{
    (x \T A) \in \ctx
  }{
    \ctx \ent x \T A
  }

  % \inferrule[Const]{
  %   (\const \T A) \in \sig
  % }{
  %   \ctx \ent \const \T A
  % }

  \inferrule[True]{
  }{
    \ctx \ent \tru \T bool
  }

  \inferrule[False]{
  }{
    \ctx \ent \fls \T bool
  }
  
  \inferrule[Fun]{
    \ctx, x \T A \ent c \T \C
  }{
    \ctx \ent \fun{x} c \T A \to \C
  }

  \inferrule[Hand]{
    \ctx, x \T A \ent c_r \T B \E \dirt \\
    \Big[
      (\op \T A_\op \to B_\op) \in \sig \qquad \\
      \ctx, y \T A_\op, k \T B_\op \to B \E \dirt \ent c_\op \T B \E \dirt
    \Big]_{\op \in \ops}
  }{
    \ctx \ent \shorthand \T \\ A \E \dirt \cup \ops \hto B \E \dirt
  }
\end{mathpar}
\textbf{Computations}
\begin{mathpar}
  \inferrule[SubComp]{
    \ctx \ent c \T \C \\
    \C \le \C'
  }{
    \ctx \ent c \T \C'
  }

  \inferrule[App]{
    \ctx \ent v_1 \T A \to \C \\
    \ctx \ent v_2 \T A
  }{
    \ctx \ent v_1 \, v_2 \T \C
  }

  \inferrule[Cond]{
    \ctx \ent v \T bool \\
    \ctx \ent c_1 \T \C \\
    \ctx \ent c_2 \T \C \\
  }{
    \ctx \ent \conditional{v}{c_1}{c_2} \T \C
  }

  \inferrule[LetRec]{
    \ctx, f \T A \to \C, x \T A \ent c_1 \T \C \\
    \ctx, f \T A \to \C \ent c_2 \T \D
  }{
    \ctx \ent \letrecin{f \, x = c_1} c_2 \T \D
  }

  \inferrule[Ret]{
    \ctx \ent v \T A
  }{
    \ctx \ent \ret v \T A \E \emptyset
  }

  \inferrule[Op]{
    (\op \T A \to B) \in \sig \\
    \ctx \ent v \T A
  }{
    \ctx \ent \op \, v \T B \E \{\op\}
  }

  \inferrule[Do]{
    \ctx \ent c_1 \T A \E \dirt \\
    \ctx, x \T A \ent c_2 \T B \E \dirt
  }{
    \ctx \ent \doin{x \leftarrow c_1} c_2 \T B \E \dirt
  }

  \inferrule[With]{
    \ctx \ent v \T \C \hto \D \\
    \ctx \ent c \T \C
  }{
    \ctx \ent \withhandle{v}{c} \T \D
  }
\end{mathpar}
\end{minipage}
}
\end{center}
\caption{Typing of \eff}\label{fig:eff-typing}
\end{figure}
