
\subsection{Typing rules}
Figure~\ref{fig:core-typing} defines the typing judgements for values and computations with respect to a standard typing context $\ctx$.

\paragraph{Values}
The rules for subtyping, variables, type abstraction, type application and functions are entirely standard. For constants we assume a signature $\sig$ that assigns a type~$A$ to each constant~$\const$, which we write as $(\const \T A) \in \sig$.

A handler expression has type $A \E \dirt \cup \ops \hto B \E \dirt$ iff all branches (both the operation cases and the return case) have dirty type $B \E \dirt$ and the operation cases cover the set of operations $\ops$. Note that the intersection $\dirt \cap \ops$ is not necessarily empty (with $\cap$ being the intersection of the operations, not to be confused with the $\intersection$ type). The handler deals with the operations $\ops$, but in the process may re-issue some of them (i.e., $\dirt \cap \ops$).

When typing operation cases, the given signature for the operation $(\op \T A_\op \to B_\op) \in \sig$ determines the type $A_\op$ of the parameter $x$ and the domain $B_\op$ of the continuation $k$. As our handlers are deep, the codomain of $k$ should be the same as the type $B \E \dirt$ of the cases.

\paragraph{Computations}
With the following exceptions, the typing judgement $\ctx \ent c : \C$ has a straightforward definition. The $\ret$ construct renders a value $v$ as a pure computation, i.e., with empty dirt. In this case, this is defined as a set with the $\dirtend$ as the only element. An operation invocation $\op\,v$ is typed according to the operation's signature, with the operation itself as its only operation. Finally, rule \textsc{With} shows that a handler with type $\C \hto \D$ transforms a computation with type $\C$ into a computation with type $\D$.

\begin{figure}[!htb]
\begin{center}
\framebox{
\begin{minipage}{0.95\columnwidth}
\[\begin{array}{r@{~}c@{~}l}
  \text{typing contexts}~\ctx & \bnfis {} & \epsilon \bnfor \ctx, x : A \bnfor \ctx, \letvar : \polytype{B}\\
\end{array}\]
\textbf{Expressions}
\begin{mathpar}
  \inferrule[Sub-Val]{
    \ctx \ent v \T A \\
    A \le B
  }{
    \ctx \ent v \T B
  }

  \inferrule[Var]{
    (x \T A) \in \ctx
  }{
    \ctx \ent x \T A
  }

  \inferrule[True]{
  }{
    \ctx \ent \tru \T bool
  }

  \inferrule[False]{
  }{
    \ctx \ent \fls \T bool
  }

  \inferrule[Fun]{
    \ctx, x \T A \ent c \T \C
  }{
    \ctx \ent \fun{x}c \T A \to \C
  }

  \inferrule[Hand]{
    \ctx, x \T A \ent c_r \T B \E \dirt \\
    \Big[
      (\op \T A_\op \to B_\op) \in \sig \qquad \\
      \ctx, x \T A_\op, k \T B_\op \to B \E \dirt \ent c_\op \T B \E \dirt
    \Big]_{\op \in \ops}
  }{
    \ctx \ent \shorthand \T \\ A \E \dirt \cup \ops \hto B \E \dirt
  }

\end{mathpar}
\textbf{Computations}
\begin{mathpar}
  \inferrule[Sub-Comp]{
    \ctx \ent c \T \C \\
    \C \le \D
  }{
    \ctx \ent c \T \D
  }

  \inferrule[App]{
    \ctx \ent v_1 \T A \to \C \\
    \ctx \ent v_2 \T A
  }{
    \ctx \ent v_1 \, v_2 \T \C
  }

  \inferrule[Cond]{
    \ctx \ent v \T A \\
    \ctx \ent c_1 \T \C \\
    \ctx \ent c_2 \T \D \\
  }{
    \ctx \ent \conditional{v}{c_1}{c_2} \T (\C \union \D)
  }

  \inferrule[Ret]{
    \ctx \ent v \T A
  }{
    \ctx \ent \ret v \T A \E \emptyrow
  }

  \inferrule[Op]{
    (\op \T A \to B) \in \sig \\
    \ctx \ent v \T A \\
    \C \T B \E \{\op \row\}
  }{
    \ctx \ent \op \, v \T \C
  }

  \inferrule[Let]{
    \ctx \ent c_1 \T A \E \dirt \\
    \ctx, x \T \polytype{A} \ent c_2 \T B \E \dirt \\
    \alpha \not\in FTV(\ctx)
  }{
    \ctx \ent \letin{\letvar = c_1} c_2 \T B \E \dirt
  }

  \inferrule[With]{
    \ctx \ent v \T \C \hto \D \\
    \ctx \ent c \T \C
  }{
    \ctx \ent \withhandle{v}{c} \T \D
  }

\end{mathpar}
\end{minipage}
}
\end{center}
\caption{Typing of \core}\label{fig:core-typing}
\end{figure}
